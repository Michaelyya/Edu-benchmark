
\documentclass{article}

\begin{document}

\title{18.01 Calculus\\
Jason Starr}
\date{Due by 2:00pm sharp Fall 2005\\
Friday, Sept. 16, 2005}
\maketitle

\textbf{Problem Set 1} 

\noindent \textbf{Late homework policy.} Late work will be accepted only with a medical note or for another Institute-approved reason.

\noindent \textbf{Cooperation policy.} You are encouraged to work with others, but the final write-up must be entirely your own and based on your own understanding. You may not copy another student's solutions. And you should not refer to notes from a study group while writing up your solutions (if you need to refer to notes from a study group, it isn't really “your own understanding”).

\section*{Part I}
These problems are mostly from the textbook and reinforce the basic techniques. Occasionally the solution to a problem will be in the back of the textbook. In that case, you should work the problem first and only use the solution to check your answer.

\section*{Part II}
These problems are not taken from the textbook. They are more difficult and are worth more points. When you are asked to “show” some fact, you are not expected to write a “rigorous solution” in the mathematician's sense, nor a “textbook solution”. However, you should write a clear argument, using English words and complete sentences, that would convince a typical Calculus student. (Run your argument by a classmate; this is a good way to see if your argument is reasonable.) Also, for the grader's sake, try to keep your answers as short as possible (but don't leave out important steps).

\subsection*{Part I (20 points)}
\begin{enumerate}
\item[(a)] (2 points) p. 57, Section 2.2, Problem 1
\item[(b)] (2 points) p. 62, Section 2.3, Problem 1
\item[(c)] (2 points) p. 62, Section 2.3, Problem 28
\item[(d)] (2 points) p. 62, Section 2.3, Problem 36 You may assume x $>$ 0. 
\item[(e)] (2 points) p. 68, Section 2.4, Problem 8 “Speed” is different from “velocity”.
\item[(f)] (2 points) p. 73, Section 2.5, Problem 4
\item[(g)] (2 points) p. 73, Section 2.5, Problem 6
\item[(h)] (2 points) p. 73, Section 2.5, Problem 14
\item[(i)] (2 points) p. 87, Section 3.1, Problem 5
\item[(j)] (2 points) p. 91, Section 3.2, Problem 30
\end{enumerate}

\subsection*{Part II (30 points)}

\noindent \textbf{Problem 1 (15 points)} The derivative of $f(x) = 1/x$ is $f'(x) = -1/x^2$ (for x nonzero).

(a) (5 points) Show that for the tangent line to the graph of $f(x)$ at $(x_0, y_0)$, the x-intercept of the line is $2x_0$ and the y-intercept of the line is $2y_0$.

\noindent Continued on next page...

\newpage

\title{18.01 Calculus\\
Jason Starr}
\date{Due by 2:00pm sharp Fall 2005\\
Friday, Sept. 16, 2005}
\maketitle

(b) (5 points) Part (a) implies the following: For every pair of real numbers $(x_1, y_1)$, there is a tangent line to the graph of $f(x)$ with x-intercept $x_1$ and y-intercept $y_1$ if and only if $x_1y_1$ equals 4. You may use this fact freely.

Let $(a, b)$ be a point such that $ab$ is nonzero and less than 1 (possibly negative). Show that a line $L$ with x-intercept $x_1$ and y-intercept $y_1$ is a tangent line to the graph of $f(x)$ containing $(a, b)$ if and only if $x_1$ satisfies,

\[bx^2_1 - 4x_1 + 4a = 0,\]

and $y_1$ satisfies,

\[ay^2_1 - 4y_1 + 4b = 0.\]

Hint: Using Equations 1–5 on pp. 11–12 of the textbook, deduce that $L$ contains $(a, b)$ if and only if $bx_1 + ay_1$ equals $x_1y_1$. Then use the fact above to eliminate one of $x_1$ or $y_1$ from this equation, and simplify.

(c) (5 points) Using the quadratic formula and Part (b), which you may now use freely, write down the equations of the 2 tangent lines to the graph of f(x) containing the point (a, b) = (5, -3).

\noindent \textbf{Problem 2 (10 points)} For a mass moving vertically under constant acceleration -g, the displacement function is,

\[x(t) = -gt^2/2 + v_{0}t + x_{0},\]

where $x_{0}$ is the displacement and $v_{0}$ is the instantaneous velocity at time $t = 0$.

A scientist uses a magnetic field to conduct an experiment simulating zero gravity. At time $t = 0$, the scientist drops a mass from a height of 10m with instantaneous velocity $v_{0} = 0$ under constant acceleration $-10m/s^2$. When the mass drops below height 5m, the field is switched on and the particles continue to move with new acceleration $0m/s^2$.

Assuming the displacement and velocity are continuous, determine the height and instantaneous velocity of the mass at time $t = 1.2s$. Show your work.

\noindent \textbf{Problem 3 (5 points)} For a differentiable function f(x) and a real number a, using the difference quotient definition of the derivative, show that the function

\[g(x) = f(ax)\]

has derivative,

\[g'(x) = af'(ax).\]

\end{document}
