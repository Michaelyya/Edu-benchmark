\documentclass{article}

\usepackage{amsmath}

\begin{document}

\begin{center}
\textbf{MASSACHUSETTS INSTITUTE OF TECHNOLOGY} \\
Physics Department \\
8.044 Statistical Physics I Spring Term 2013 \\
Problem Set \#5 \\
Due in hand-in box by 12:40 PM, Monday, March 11 
\end{center}

\noindent \textbf{Problem 1: Correct Boltzmann Counting} 

\noindent The calculation we have done so far to obtain the allowed volume in phase space, $\Omega$, for a 
classical system is in error. We will demonstrate the results of this error in two different 
cases and then propose a remedy. 

\noindent a) A state variable F is extensive if, after multiplying all the extensive variables in the expression for F by a scale factor $\lambda$ and leaving all the intensive variables in F unchanged, 
the result is a $\lambda$ fold increase in F , that is, $\lambda F$. The expression we found for the cumulative volume in phase space for an ideal monatomic gas using the microcanonical 
ensemble was 
$\Phi(E, V, N) = \frac{V^{N} \space 4 \pi e^{mE}}{3^{N} \space N^{3N/2}}$.
Use $S = k \ln \Phi$ and the derived result $E = \frac{3}{2} NkT$ to write S as a function of the 2 thermodynamic variables N, V , and T . On physical grounds S should be extensive. 
Show that our expression for $S(N, V, T)$ fails the above test for an extensive variable. 

\noindent b) Consider a mixing experiment with two ideal gases, 1 and 2. A volume V is separated 
into two parts $V_{1} = \alpha V$ and $V_{2} = (1 - \alpha) V$ by a movable partition $0 \leq \alpha \leq 1$. Let 
$N_{1}$ atoms of gas 1 be confined in $V_{1}$ and $N_{2}$ atoms of gas 2 occupy $V_{2}$. Show that if 
the temperature and pressure are the same on both sides of the partition, the ideal 
gas equation of state requires that $N_{1} = \alpha N$ and $N_{2} = (1 - \alpha)N$ where $N = N_{1} + N_{2}$. 
Pulling the partition out allows the gases to mix irreversibly if the gases are different. 
The mixing is irreversible but entropy is a state function so $\Delta S_{i} = S_{\text{final}} - S_{\text{initial}}$ 
can be computed for each gas from the expression in a). Show that 

\begin{align*}
\Delta S_{1} &= \alpha Nk \ln(1/\alpha) \\
\Delta S_{T} &= \Delta S_{1} + \Delta S_{2} = Nk [\alpha \ln(1/\alpha) + (1 - \alpha) \ln(1/(1 - \alpha))] 
\end{align*}

\noindent This expression for the entropy of mixing is always positive, which is the result we 
expect based upon the disorder interpretation of entropy. \\
Should $S_{T}$ increase as we slide the partition out when the two gases are the same? This 
is difficult to answer from an intuitive point of view since the presence of the partition 
does restrict the atomic motion. 

\noindent Macroscopic thermodynamics, however, requires that $\Delta S_{T} = 0$ in this case. 

\end{document}