\documentclass{article}
\usepackage{amsmath}
\usepackage{amsfonts}
\usepackage{amssymb}

\title{MIT Physics Department
8.044 Statistical Physics I\\Spring Term 2013\\Problem Set \#9}
\date{Due in hand-in box by 4:00 PM, Friday April 19th}

\begin{document}
\maketitle
\section*{Problem 1: The Big Bang}
Early in the evolution of the universe, when the universe occupied a much smaller volume and was very hot, matter and radiation were in thermal equilibrium. However, when the temperature fell to about 3000K, matter and the cosmic radiation became decoupled. The temperature of the cosmic (black body) radiation has been measured to be 3K now. The free energy of thermal radiation is $F(T,V) = -\frac{1}{3I_3}\frac{\pi^2}{45 c}(kT)^4V$.

\section*{Problem 2: Lattice Heat Capacity of Solids}
A crystalline solid is composed of $N$ primitive unit cells, each containing $J$ atoms. A primitive unit cell is the smallest part of the solid which, through translational motions alone, could reproduce the entire crystal. The atoms in the unit cell could be the same or they could differ: diamond has two carbon atoms per primitive unit cell, sodium chloride has one Na and one Cl. \\

\noindent\textbf{2a)} The Classical Model. \\
\noindent\textbf{2b)} The Einstein Model. \\
\noindent\textbf{2c)} Phonons. The density of frequencies $D(\omega)$ is defined such that $D(\omega_0) d\omega$ is the number of phonons in the crystal with frequencies between $\omega_0$ and $\omega_0 + d\omega$. Normalization requires that $\int_{0}^{\infty} D(\omega) d\omega = 3JN$.

\section*{Problem 3: Thermal Noise in Circuits I, Mean-Square Voltages and Currents}
An arbitrary network of passive electronic components is in thermal equilibrium with a reservoir at temperature $T$. It contains no sources.

\section*{Problem 4: Thermal Noise in Circuits II, Johnson Noise of a Resistor}
When we discussed jointly Gaussian random variables in the first part of this course, we learned that the noise voltage in a circuit is a random process, a signal which evolves in time. It will be composed of a variety of different frequency components. The noise power in a unit frequency interval centered at radian frequency $\omega$, $S_v(\omega)$, is referred to as the power spectrum, or simply the spectrum, of the voltage fluctuations.

\section*{Problem 5: Thermal Noise in Circuits III, Circuit Model for a Real Resistor}
Now we will see how the concept of Johnson noise associated with a resistor is used to find real noise voltages in a circuit.

\end{document}