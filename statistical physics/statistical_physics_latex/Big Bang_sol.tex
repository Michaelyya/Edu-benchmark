\documentclass{article}
\usepackage{amsmath}
\title{\textbf{MASSACHUSETTS INSTITUTE OF TECHNOLOGY\\Physics Department\\8.044 Statistical Physics I Spring Term 2013\\Solutions to Problem Set \#9}}

\begin{document}
\maketitle
\noindent
\textbf{Problem 1: The Big Bang}

If the expansion is adiabatic, $\Delta S = 0$.

$S=\beta^2\partial F/\partial T - \beta V \text{ and } \partial /\partial T (kT)^4V/45c3 =~3/ 4(kT)^2=\beta k^4T^3V/45c3~^3$

From this result we see that the product $T^3V$ remains constant during the expansion. Therefore,

$V(3K)/V(3000K) = 3000^3/3^3 = 10^9

\textbf{Problem 2: Lattice Heat Capacity of Solids}

a) The heat capacity of a classical harmonic oscillator is $k$, independent of its frequency.

\# of oscillators = (3 degrees of freedom)  (J atoms/unit cell) (N unit cells) = $3 JN$. 

Therefore, $C_V= 3JNk$.

b) For a quantum harmonic oscillator

$1<\psi> =h\omega(<n> + )^2$

$<n> = (exp[h\omega=kT ]-1)^{-1}$

... ...

\textbf{Problem 3: Thermal Noise in Circuits I, Mean-Square Voltages and Currents}

... ...

\textbf{Problem 4: Thermal Noise in Circuits II, Johnson Noise of a Resistor}

... ...

\textbf{Problem 5: Thermal Noise in Circuits III, Circuit Model for a Real Resistor}

... ...

THE THERMAL NOISE IN A CIRCUIT CAN BE THOUGHT OF (AND QUANTITATIVELY MODELED) AS ARISING FROM THE DISSIPATIVE ELEMENTS IN THE SYSTEM.

\end{document}