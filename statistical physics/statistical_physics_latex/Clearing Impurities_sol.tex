\documentclass{article}
\usepackage[utf8]{inputenc}
\usepackage{amsmath}

\begin{document}

\title{MASSACHUSETTS INSTITUTE OF TECHNOLOGY \\
Physics Department\\
8.044 Statistical Physics I Spring Term 2013\\
Solutions to Problem Set \#3}
\maketitle

\section{Problem 1: Clearing Impurities}

Since we are asked for an approximate answer we will resort to the central limit theorem.
For this we need $<x>$ and $<x^2>$ for a single sweep of the laser beam.

\begin{align*}
<x> &= \int x p(x) dx = a \int Z^2 e^{-Z} dZ = a \cdot \frac{1}{3} |_{0}^{1} \\
<x^2> &= \int Z^{14} x^2   p(x) dx= a^2 \cdot \frac{3}{Z} \int Z^2 e^{0-Z} dZ = a^2 \cdot \frac{1}{|} ^{ }_{2}3 \\
Var(x) &= <x^2> - <x>^2 = \frac{3}{2^4} - \frac{4}{4} = \frac{a^2}{8} - \frac{a^2}{9}
\end{align*}

The general form of the central limit theorem is

\begin{equation*}
\frac{1}{p(d)} e^{- \frac{(d - <d>)^2}{2 \sigma^2}}
\end{equation*}

with

\begin{align*}
<d> &= 36 \\
2<x> &= 24a \\
\sigma &= 36 Var(x) = 32 a^2
\end{align*}

Although it was not asked for, here is a sketch of the resulting probability density.
\section{Problem 2: Probability Densities of Macroscopic verses Microscopic Variables}

a) Let $E_1$ be the kinetic energy of a single atom in the gas. We can begin with the expression
for $p(E_1)$ found in problem 4 on Problem Set 2.


\end{document}