```
\documentclass{article}
\usepackage{amsmath}
\usepackage{amssymb}

\begin{document}

\title{Massachusetts Institute of Technology \\
Physics Department \\
8.044 Statistical Physics I Spring Term \\
Solutions to Problem Set \#7}

\maketitle

\noindent \textbf{Problem 1: Free Expansion of a Gas}

\noindent \textbf{a)} No work is done, so $\Delta W = 0$. No heat enters the gas so $\Delta Q= 0$. Thus $\Delta E = \Delta W+ \Delta Q = 0$. The internal energy is conserved.

\noindent \textbf{b)} $E(T;V )$ is a state function; compare it before and after expansion in equilibrium situations.

\begin{equation*}
\begin{split}
dE &= TdS + PdV = T \left (\frac{\partial S}{\partial T} \right )_V dT + V  \left (\frac{\partial P}{\partial T} \right )_V dV \\
\end{split}
\end{equation*}

\setcounter{section}{1}
\section*{Problem 2: Use of a Carnot Cycle}

\noindent \textbf{a)} Carnot Cycle $dS_H= \frac{dQ_H}{T_H}$, $dS_C = \frac{dQ_C}{T_C}$. 

\noindent Used $Q=C_0dT$:

\begin{equation*}
\begin{split}
\Delta S_{\text{body 1}} &= -\Delta S_{\text{body 2}} \\
&= -\int_{T_F}^{T_H} \frac{C_0}{T}dT \\
&= -C_0\left[ \ln\left(\frac{T_H}{T_F}\right) - \ln\left(\frac{T_C}{T_F}\right)\right] \\
&= C_0\left [\ln\left(\frac{T_F}{T_C}\right) - \ln\left(\frac{T_H}{T_F}\right) \right ] \\
&= C_0\ln\left( \frac{T_F^2}{T_HT_C}\right) \\
\end{split}
\end{equation*}

\setcounter{section}{4}
\section*{Problem 5: The Hydrogen Atom}

\begin{equation*}
\begin{split}
A |n;l;m \rangle &= -\frac{1}{n^2} |n;l;m \rangle  \\
\end{split}
\end{equation*}

\noindent \textbf{c)} The Coulomb potential is a mathematical oddity in that it produces an infinite number of bound states with energies less than zero. This situation is modified in the real world by the presence of walls (consider the energy levels of a particle in a box) or by the presence of other atoms. The existence of hydrogen atoms in the interstellar medium, on the other hand, probably has more to do with the absence of excitation mechanisms (non-equilibrium) than with the presence of neighboring atoms.

\end{document}
```
Keep in mind that I just converted some parts of the text to LaTeX, there might be parts that needed to be converted that I missed.