\documentclass{article}

\begin{document}

\title{MASSACHUSETTS INSTITUTE OF TECHNOLOGY\\ Physics Department\\ 8.044 Statistical Physics I Spring Term 2013\\ Problem Set \#7}
\date{Due in hand-in box by 12:40 PM, Wednesday, April 3}
\maketitle

\noindent \textbf{Problem 1: Free Expansion of a Gas}

Before After

\noindent A classical, monatomic, non-ideal gas has the equation of state

\begin{equation}
P (T, V ) = \frac{{NkT}}{{V − bN}} − \frac{{aN}}{{V^2}}
\end{equation}

\noindent where $a$ and $b$ are positive constants.

\begin{enumerate}
	\item Which thermodynamic quantities are conserved in this process?
	\item What is the new temperature of the gas when thermal equilibrium has finally been re-established?
\end{enumerate}

\noindent \textbf{Problem 2: Use of a Carnot Cycle}

\begin{enumerate}
	\item Find $T_F$ in terms of $T_H$ and $T_C$.
	\item Find the total work done on the outside world in this process. Is it positive or negative?
\end{enumerate}

\noindent \textbf{Problem 3: Cooling Liquid Helium}

\begin{enumerate}
	\item What must be the initial temperature of the salt, $T_0$, if the final temperature of the system, helium plus salt, is to be 0.5 K?
	\item Find the net change in the entropy of the universe during this process. To facilitate grading, and to minimize the chance of math errors, leave this answer in terms of the parameters $a$ and $b$.
\end{enumerate}

\noindent \textbf{Problem 4: Torsional Pendulum}

\begin{enumerate}
	\item Find the root-mean-square uncertainty $< (\theta-\theta_0)^2 >^{1/2}$ associated with a measurement of $\theta$ due to classical thermal noise.
	\item Find $<\theta \dot{\theta}>$.
\end{enumerate}

\noindent \textbf{Problem 5: The Hydrogen Atom}

\begin{enumerate}
	\item Find the ratio of the number of atoms in the first excited energy level to the number in the lowest energy level at a temperature $T$. [Hint: Be careful of the difference between energy levels and states of the system.] Evaluate this for $T = 300K$ and $T = 1000K$.
	\item To find the actual fraction of atoms in any given state one needs the partition function. Show that the partition function diverges, even when the unbound states are neglected.
	\item Any ideas why statistical mechanics does not seem to work for this, one of nature’s simplest systems?
\end{enumerate}

\end{document}