\documentclass{article}
\usepackage{amsmath}

\begin{document}

\title{Fall 2018 14.01 Problem Set 5 -  Solutions}
\maketitle

\section*{Problem 1: True or False (24 points)}
Determine whether the following statements are True or False. Explain your answer.

\begin{enumerate}
\item (4 points) A government sets a price ceiling for widgets that is below the equilibrium price. This intervention will always decrease the producer surplus, increase consumer surplus and decrease total surplus.

\textbf{Solution:} False, it might increase or decrease consumer surplus. Price decreases but also quantity decreases. The firm(s) might also exit the market if the price is below average cost which would result in zero consumer surplus.

\item (4 points) A non-discriminating monopoly will optimally choose a price and quantity where the market demand curve is inelastic (i.e. elasticity is lower than one in absolute value).

\textbf{Solution:} False, the monopolist will always choose a price where the demand is elastic.

\item (4 points) If a monopolist charging a uniform price begins to practice perfect price discrimination, some consumers may become strictly better off.

\textbf{Solution:} False, the perfectly discriminating monopolist will extract all the surplus from consumers, so no consumer can be strictly better off.

\item (4 points) Transferring a monopolist’s profit to consumers eliminates the inefficiency associated with monopoly.

\textbf{Solution:} False. The total surplus will not change. Because this policy would increase consumer surplus and reduce producer surplus by the same amount. There is still a deadweight loss associated with a monopolist producing a quantity that is lower than the socially optimal given by p = MC.

\item (4 points) Martin knows that if he sells 500 widgets, his revenue will be \$1000 and that if he sells 800 widgets, his revenue will be \$1500. The market for widgets is perfectly competitive.

\textbf{Solution:} False. Martin’s output affects the market price which indicates that he is not a price taker.

\item (4 points) Suppose that a monopolist initially charging a uniform price, now moves to charging a different price to young and old consumers, but the total quantity sold by the monopolist doesn’t change. Claim: total welfare must be lower after the monopolist start discriminating.

\textbf{Solution:} True, if the quantity sold is the same then total costs are the same, but because with price discrimination the goods might not be going to those who value them the most, the allocation becomes more inefficient. Note: don’t take down points if students answer that the monopolist might choose not to discriminate when she is allowed to, so total welfare is unchanged (that is, the monopolist  might choose to sell at a uniform price even if she can discriminate between old and young consumers).
\end{enumerate}

\section*{Problem 2 (41 points)}
Consider the perfectly competitive market for gasoline. The aggregate demand for gasoline is 
\[D (p) = 100 - p\]
while the aggregate supply is 
\[S (p) = 3p\]
\begin{enumerate}
\item (5 points) Calculate the equilibrium price and quantity. At this equilibrium, compute the consumer surplus, producer surplus and total surplus.

\textbf{Solution:} 
\begin{align*}
p &= 25\\
q &= 75\\
CS &= \frac{75^2}{2} = 2812.5\\
PS &= \frac{25 × 75}{2} = 937.5\\
TS &= 2812.5 + 937.5 = 3750
\end{align*}

\item (5 points) Suppose now that the government is concerned because many gas stations are going out of business, so it decides to set a minimum price $\bar{p} = 30$ to help them. What will be the new equilibrium price and quantity with this intervention? Compute the consumer surplus and producer surplus; who gains and who lose from this regulation? How is the total surplus affected? Briefly explain the intuition.

\textbf{Solution:} The price will be $p = 30$, and the quantity $q = 70$. Consumers are worse off, producers are better off, and the total surplus decreases.

\item (5 points) Suppose now that instead of regulating prices, the government decides it is better to help gas stations by setting quantity regulations. In particular the government sets a quota $\bar{q} = 70$ (this means that aggregate quantity supplied can’t exceed 70 units). What will be the new equilibrium price and quantity with this regulation? How does it compare to the results obtained with the minimum price?


\textbf{Solution:} We will obtain the same equilibrium as with the minimum price, so both policies are equivalent.
\end{enumerate}

\end{document}
