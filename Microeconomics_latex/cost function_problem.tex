\documentclass{article}
\usepackage{amsmath}
\begin{document}

\title{Fall 2018 14.01 Problem Set 3}
\maketitle

\section*{Problem 1 (28 points)} 

True/False/Uncertain. Please fully explain your answer, including diagrams where appropriate. Points are awarded based on explanations.

\begin{enumerate}
\item (4 points) The short run marginal cost curve is always increasing due to the law of diminishing marginal returns.
\item (4 points) In the short run, perfectly competitive firms with higher fixed costs must also charge a higher price, all else equal.
\item (4 points) If marginal cost is increasing, then average cost must increase as well.
\item (4 points) If a firm’s production function exhibits decreasing marginal returns in each factor, then it must also exhibit decreasing returns to scale.
\item (4 points) Two firms in the same perfectly competitive market, A and B, have short run costs given by $CA(q) = 10 + 2q^2$ and $CB(q) = 10 + 3q^2$ respectively. Since B has higher costs, it must charge a higher price in equilibrium.
\item (4 points) If a firm has U-shaped (convex) marginal cost, then AVC and MC are equal at the point where AVC is minimized.
\item (4 points) A firm with production function $q = L + 2K$ will always hire either labor or capital, but never both at the same time.
\end{enumerate}
\newpage

\section*{Problem 2 (16 points)}

A firm has a production function $q = f(K, L) = K + 0.5L$ and faces wages $w$ and rental rate of capital $r$. Let $w = 1$, $r = 1$. For this problem, think about the long-run where capital is not fixed.

\begin{enumerate}
\item (8 points) Suppose the firm wants to produce $q = 200$. What is the combination of K and L that minimizes total cost? Draw the isoquant and isocost curves that correspond to the firm’s optimal choice, with K in the y-axis and L in the x-axis. Explain.
\item (8 points) Suppose that the government wants to encourage the use of labor and decides to pay for 50\% of the firm’s wage costs for the first 100 units of labor used – i.e. the firm only pays $0.5w$ for the first 100 units of labor used. Again, assume that the firm wants to produce $q = 200$. What are all the possible combinations of K and L that minimize total cost? Draw the isoquant and isocost curves that correspond to the firm’s optimal choice. Explain.
\end{enumerate}

\section*{Problem 3 (16 points)}

In the short run, a firm has fixed capital $K$. We know that its short-run cost function is $CSR (q) = q^3 - 2q^2 + 2q + 2$.

\begin{enumerate}
\item (8 points) Plot the short-run marginal cost and average variable cost (as a function 
of $q$). What is the short-run supply curve?
\item (8 points) Suppose that the long-run cost curve is $CLR (q) = 23 q^2$. Can you find 
the quantity $\overline{q}$ such that in the long-run the firm optimally chooses to use $K$ units 
of capital to produce $\overline{q}$? (Hint: if you are having trouble finding the solution by 
hand, use a numerical solver; Wolfram Alpha is a great resource!)
\end{enumerate}

\newpage

\section*{Problem 4 (40 points)}

This is a somewhat mathematically involved problem. Please show your work, partial credit will be given.

A firm has a Cobb-Douglas production function $q = f(K, L) = K^\alpha L^{1-\alpha}$ and faces wages, $w$, and rental rate of capital, $r$.

\begin{enumerate}
\item (3 points) Does this production function exhibit increasing, decreasing, or constant returns to scale?
\item (6 points) Find the short-run cost curve, $C(q)$, as a function of $q$ and the parameters.
\item (6 points) For this subpart only, assume that $\bar{K} = 10$, $r = 1.5$, $w = 6$, and $\alpha = 2/3$. Derive expressions for MC, VC, FC, ATC, AVC, and AFC. Plot MC, ATC, AVC, and AFC, all on the same graph (using a graphing program – WolframAlpha, Mathematica, Matlab, etc.– is fine for this part).
\item (6 points) For this subpart only , assume that $\bar{K} = 10$, $r = 1.5$, $w = 6$, and $\alpha = 2/3$. Assume now that we know the market price is $p = 18$, which is fixed,
and we are still operating in the short-run. What is the profit-maximizing choice of $q$?
\item (6 points) Solve for profits, $\pi$, as a function of market price, $p$ (and the parameters $\bar{w}$, $r$, $\bar{K}$, $\alpha$). Then, assume as we did in subpart 3 that $\bar{K} = 10$, $r = 1.5$, $w = 6$, and $\alpha = 2/3$, and continue to assume so until subpart 6 (included). Will profits ever be negative? If so, find the price range at which profits are negative.
\item (6 points) Is the firm better off shutting down (producing $q = 0$) at any positive low price? Explain.
\item (7 points) Going back to the original production function, let’s think about the firm’s problem in the long run. Find their optimal choices of inputs (as a function 
of $q$, $w$, $r$, $\alpha$) and the resulting long-run cost function $C(q)$. How does the firm’s choice of $L$ and $K$ change if $r$ goes up? What happens to costs?
\end{enumerate}

\end{document}