\documentclass{article}

\begin{document}

\section{Supply and demand}

\subsection{Lecture 2: Supply and Demand}

\subsubsection{Supply and demand diagrams:}

\begin{itemize}
\item Demand Curve measures willingness of consumers to buy the good 
\item Supply Curve measures willingness of producers to sell 
\item Intersection of supply and demand curve is market equilibrium. 
\item Supply and demand curves can shift when there are:
\begin{itemize}
\item shocks to the ability of producers to supply 
\item shocks in consumer tastes 
\item shocks to the price of complement/substitute goods. A rise in the price of a substitute good for good X raises the demand for the X. 
\end{itemize}
\item Interventions in market can lead to disequilibrium:
\begin{itemize}
\item for example, imposing a minimum wage means that more people will want to work than employers want to hire at the minimum wage. This creates unemployment. 
\end{itemize}
\item The cost of these interventions is found in reduced efficiency (trades that are not made); there may be benefits in greater equity. 
\end{itemize}

\subsubsection{TO KNOW- Conceptual Understanding}
\begin{itemize}
\item Explain the difference between a movement along the demand (supply) curve and a shift of the demand (supply) curve 
\item Describe factors that shift supply and demand curves 
\item Know "what's wrong" with excess supply or excess demand 
\end{itemize}

\subsubsection{TO KNOW- Graphical and Math Understanding}
\begin{itemize}
\item Find a market equilibrium given a demand and supply curve- (a) graphically and (b) using algebraic expressions 
\item Analyze the effect of a price ceiling in a graph 
\item Analyze the effect of a price floor in a graph 
\end{itemize}

\subsection{Lecture 3: Applying supply and demand}
\subsubsection{Elasticity}

\begin{itemize}
\item Price elasticity of demand is defined: $\epsilon = \frac{\partial Q}{\partial P} \cdot \frac{Q}{P}$
\item Perfectly inelastic demand is $\epsilon = 0$ and perfectly elastic demand is $\epsilon = -\infty$.
\item The elasticity affects consumers' response to a shift in price: if the elasticity is between 0 and -1, then firms can raise revenues by raising the price (since consumers will still buy the good in significant quantities); if $\epsilon < -1$, ...
\item ...
\end{itemize}

\end{document}
