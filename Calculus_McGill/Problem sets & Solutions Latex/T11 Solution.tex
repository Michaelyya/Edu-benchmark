\documentclass[10pt]{article}
\usepackage[utf8]{inputenc}
\usepackage[T1]{fontenc}
\usepackage{amsmath}
\usepackage{amsfonts}
\usepackage{amssymb}
\usepackage[version=4]{mhchem}
\usepackage{stmaryrd}

\begin{document}
\begin{enumerate}
  \item Sketch the graphs of a continuous functions on the interval $[1,5]$ and which satisfies the following properties\\
(a) Absolute maximum at 1, absolute minimum at 3, local minimum at 2 and local maximum at 4.\\
(b) Absolute maximum at 2, absolute minimum at 5, 4 is a critical number but there is no local maximum or minimum there.
  \item Find the absolute maximum and absolute minimum values of $f$ on the given interval.\\
(a) $f(x)=x^{3}-6 x+5,[-2,5]$\\
(b) $f(x)=x+\frac{1}{x},[0.2,4]$\\
(c) $f(x)=x^{a}(1-x)^{b},[0,1]$ ( $a$ and $b$ are positive, real numbers)\\
(d) $f(t)=2 \cos t+\sin 2 t,[0, \pi / 2]$
\end{enumerate}

\section*{Solution}
(a) First we find the critical points by solving the equation $f^{\prime}(x)=0$. This is

$$
3 x^{2}-6=0
$$

Which has two solutions $\pm \sqrt{2}$. Since $f$ is continuous, the absolute maximum and minimum will either be critical points or will be on the boundary of the interval. So we have to check all four points.

$$
f(-2)=9, f(-\sqrt{2})=10.657, f(\sqrt{2})=-0.657, f(5)=100
$$

Therefore, the absolute maximum is 100 and the absolute minimum is -0.657\\
(b) Again we find the critical points by solving $f^{\prime}(x)=0$. This is

$$
1-\frac{1}{x^{2}}=0
$$

, which has two solutions $\pm 1$, only one of which is on our interval. Since $f$ is continuous, the absolute maximum and minumum will either be critical points or will be on the boundary. So we have to check all three points.

$$
f(0.2)=5.2, f(1)=2, f(4)=4.25
$$

Therefore the absolute maximum is 5.2 and the absolute minimum is 2 .\\
(c) We solve the equation $f^{\prime}(x)=0$. This is

$$
a x^{a-1}(1-x)^{b}-b x^{a}(1-x)^{b-1}=0
$$

We can take $x^{a-1}(1-x)^{b-1}$ as a common factor and divide by it to get ( note that $x^{a-1}(1-x)^{b-1}$ is equal to 0 only at $x=0,1$ which are our boundary points.)

$$
a(1-x)-b x=0
$$

The solution to this equation is $x=\frac{a}{a+b}$. We can conclude that the absolute maximum is

$$
\frac{a^{a} b^{b}}{(a+b)^{a+b}}
$$

and the absolute minimum is 0 .\\
(d) As before, we will solve $f^{\prime}(t)=0$. This is

$$
-2 \sin t+2 \cos 2 t=0
$$

Now we can use the trig identity

$$
\cos 2 t=1-2 \sin ^{2} t
$$

. We get the equation

$$
2 \sin ^{2} t+\sin t-1=0
$$

This is a quadratic equation in $\sin t$, which we can solve. It has solutions $\sin t=$ $\frac{1}{2},-1$. Therefore, in the interval $[0, \pi / 2]$ the only critical point is at $t=\pi / 6$. To find the absolute maximum and minimum we just need to check the three points

$$
f(0)=2, f(\pi / 6)=2.598, f(\pi / 2)=0
$$

Therefore, the absolute max is 2.598 and the absolute min is 0.\\
3. Prove that the function

$$
f(x)=x^{101}+x^{51}+x+1
$$

has neither a local maximum or a local minimum.

\section*{Solution}
The function is differentiable, therefore the critical points will correspond to points where $f^{\prime}(x)=0$. We know that

$$
f^{\prime}(x)=101 x^{100}+51 x^{50}+1
$$

This is never 0 , because $x^{1} 00=\left(x^{50}\right)^{2}$ which is always greater than or equal to 0 , and similarly $x^{50}=\left(x^{25}\right)^{2}$.\\
4. Find two numbers whose difference is 10 and whose product is minimal.

\section*{Solution}
We will label the two numbers $a, b$, such that $a-b=10$. We want to minimize $a b$. Note that we can write $a=10+b$. So $a b=(10+b) b$. Therefore, we are trying to minimize the function $f(b)=(10+b) b$. First, we check the critical points $f^{\prime}(b)=b+(b+10)=2 b+10$. So the only critical point is at $b=-5$. This is a local minimum. This can also be seen by completing the square

$$
(10+b) b=(b+5)^{2}-25
$$

which has a minimum at $b=-5$.\\
5. Which point on the line

$$
y=2 x+1
$$

is closest to the origin?

\section*{Solution}
For a given point $(x, y)$ on the line, its distance to the origin is $\sqrt{x^{2}+y^{2}}$. Therefore, we are trying to minimize the value of $x^{2}+y^{2}$. By plugging in the equation for the line, we find that we are trying to minimize the value of

$$
f(x)=x^{2}+(2 x+1)^{2}=5 x^{2}+4 x+1
$$

We can again find the critical point by differentiation

$$
f^{\prime}(x)=10 x+4
$$

, therefore the minimum will be at $x=\frac{-2}{5}, y=\frac{1}{5}$. Once again, this can be seen by completing the square

$$
5 x^{2}+4 x+1=5\left(x+\frac{2}{5}\right)^{2}+\frac{1}{5}
$$

\begin{enumerate}
  \setcounter{enumi}{5}
  \item Find the points on the ellipse $4 x^{2}+y^{2}=4$ which are farthest from the point $(0,1)$.
\end{enumerate}

\section*{Solution}
Given coordinates $(x, y)$, the square of the distance to the point $(0,1)$ is

$$
x^{2}+(y-1)^{2}
$$

This is the function we are trying to maximize. We can plug in the constraint of the ellipse, which is

$$
x^{2}=1-\frac{y^{2}}{4}
$$

So, we are trying to maximize

$$
1-\frac{y^{2}}{4}+y^{2}-2 y+1
$$

which is proportional to

$$
3 y^{2}-8 y+5
$$

We are trying to maximize this function on the interval

$$
-2 \leq y \leq y
$$

because we need $4 x^{2}+y^{2}=4$. To find the critical points we need to differentiate, and solve the equation

$$
6 y-8=0
$$

which has solution

$$
y=\frac{4}{3}
$$

Now we want to figure out the $x$-coordinate. They satisfy $4 x^{2}+y^{2}=4$, plugging in $y=\frac{4}{3}$ gives

$$
x^{2}=\frac{8}{9}
$$

so

$$
x= \pm \frac{2 \sqrt{2}}{3}
$$

The two critical points

$$
\left( \pm \frac{2 \sqrt{2}}{3}, \frac{4}{3}\right)
$$

Now we need to compare the critical points to the boundary points, which are $(0, \pm 2)$ by comparing these points we see that the one that is furthest is $(0,-2)$.\\
7. A cylindrical can is made from material such that the top and bottom cost twice as much as the sides. What dimensions will minimize the cost of a 4 L can?

\section*{Solution}
A cylinder has height $h$, and radius $r$. Its volume is

$$
\pi r^{2} h=4
$$

The cost of the material is

$$
4 \pi r^{2}+2 \pi r h
$$

So we are trying to minimize

$$
2 r^{2}+r h
$$

we can also plug in the constraint $h=\frac{4}{\pi r^{2}}$. So we are trying to minimize

$$
2 r^{2}+\frac{4}{\pi r}
$$

where $r$ lies in the interval

$$
0<r
$$

We differentiate and set to 0 , to get the equation $4 r-\frac{4}{\pi r^{2}}=0$ which is equivalent to $r^{3}=\frac{1}{\pi}$. So $r=\pi^{-1 / 3}$, therefore $h=4 \pi^{-1 / 3}$.\\
8. A person is standing at coordinates $(1,4)$. They are making their way back to their house which has coordinates $(6,8)$, but first they need to pass by a river, which goes down the $x$-axis. What is the fastest path for this trip? At what $x$-coordinate should this person reach the river?

\section*{Solution}
There are several methods to solve this problem. I will present two, the first one being a longer tedious solution and the second is slightly simpler.

\section*{Solution 1}
Clearly the fastest path would be a straight line to some point $(x, 0)$ on the $x$-axis and then a straight line back to the house at $(6,8)$. The total distance is then

$$
\sqrt{(x-1)^{2}+4^{2}}+\sqrt{(x-6)^{2}+8^{2}}
$$

which is the function we are trying to minimize. To do so, we differentiate and equate it to 0 . This gives

$$
\frac{x-1}{\sqrt{(x-1)^{2}+4^{2}}}+\frac{x-6}{\sqrt{(x-6)^{2}+8^{2}}}=0
$$

which is the same as

$$
(x-1) \sqrt{(x-6)^{2}+8^{2}}+(x-6) \sqrt{(x-1)^{2}+4^{2}}=0
$$

by moving things around and squaring we get

$$
\frac{(x-1)^{2}}{(x-6)^{2}}=\frac{(x-1)^{2}+4^{2}}{(x-6)^{2}+8^{2}}
$$

This is equivalent to

$$
4(x-1)^{2}=(x-6)^{2}
$$

which has positive solution $x=\frac{8}{3}$, which is the critical point. Therefore, the fastest path is to go straight to the point $\left(\frac{8}{3}, 0\right)$, and then straight to the house.

\section*{Solution 2}
As noted before, the fastest path must be a straight line to the river, and then a straight line to the house. Notice that if the house had coordinates $(6,-8)$, then the fastest path to the house is exactly the line from $(1,4)$ to $(6,-8)$ and this line passes through the $x$-axis at $x=\frac{8}{3}$. Notice that the fastest path then to the house would be to go to the river at $\left(\frac{8}{3}, 0\right)$ and then reflect the line which goes to $(6,-8)$.


\end{document}