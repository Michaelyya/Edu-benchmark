\documentclass[10pt]{article}
\usepackage[utf8]{inputenc}
\usepackage[T1]{fontenc}
\usepackage{amsmath}
\usepackage{amsfonts}
\usepackage{amssymb}
\usepackage[version=4]{mhchem}
\usepackage{stmaryrd}

\begin{document}
\begin{enumerate}
  \item For the following exercises you are given $\frac{d f}{d x}$. Can you come up with some function $f(x)$ such that its derivative is the given $\frac{d f}{d x}$\\
(a) $\frac{d f}{d x}=x^{3}+x+1$\\
(b) $\frac{d f}{d x}=\sin x$\\
(c) $\frac{d f}{d x}=e^{x+2}+\frac{x}{2}$
\end{enumerate}

\section*{Solution}
(a)

$$
f(x)=\frac{x^{4}}{4}+\frac{x^{2}}{2}+x
$$

(b)

$$
f(x)=\cos x
$$

(c)

$$
f(x)=e^{x+2}+x^{2}
$$

\begin{enumerate}
  \setcounter{enumi}{1}
  \item Find the most general antiderivative.\\
(a) $f(x)=0$\\
(d) $y(\theta)=\cos (\theta)-\sin (\theta)$\\
(b) $f(x)=3 x^{3}+2 x^{2}+x+1$\\
(e) $f(x)=5 e^{x}-3 \cosh x$\\
(c) $h(y)=17 e^{-2 y}+123 \sec ^{2} x$\\
(f) $g(t)=\sin t+2 \sinh t$
\end{enumerate}

\section*{Solution}
(a)

$$
g(x)=c
$$

where $c$ is any constant real number\\
(b)

$$
g(x)=\frac{3}{4} x^{4}+\frac{2}{3} x^{3}+\frac{1}{2} x^{2}+x+c
$$

where $c$ is any constant real number\\
(c)

$$
g(x)=123 \tan x-\frac{17}{2} e^{-2 y}+c
$$

where $c$ is any constant real number\\
(d)

$$
g(\theta)=\sin \theta+\cos \theta+c
$$

where $c$ is any constant real number\\
(e)

$$
g(x)=5 e^{x}-3 \sinh x+c
$$

where $c$ is any constant real number


\begin{equation*}
h(t)=-\cos t+2 \cosh t+c \tag{f}
\end{equation*}


where $c$ is any constant real number\\
3. Find a function $f$ which satisfies the given conditions.\\
(a) $f^{\prime \prime}(x)=6 x+12 x^{2}$\\
(b) $f^{\prime \prime}(x)=2 e^{t}+3 \sin t$ with $f(0)=f(\pi)=0$

Parts (c) and (d) are more difficult that usual, and are certainly more difficult than questions to come on the final exam.\\
(c) $f^{\prime}(x)=f(x)$ with $f(0)=1$\\[0pt]
[hint: Try to re-write this equation in terms of the function $\left.g(x)=e^{-x} f(x)\right]$\\
(d) $f^{\prime \prime}(x)=f(x)$ with $f(0)=2$ and $f^{\prime}(0)=0$\\
$\left[\right.$ Try writing $g(x)=e^{x} f(x)$ and show that $g$ satisfies the equation

$$
g^{\prime \prime}(x)=2 g^{\prime}(x)
$$

Then write an equation in terms of the function $\left.h(x)=e^{-2 x} g^{\prime}(x)\right]$

\section*{Solution}
(a) By taking the anti-derivative we know that

$$
f^{\prime}(x)=3 x^{2}+4 x^{3}+c_{1}
$$

for some constant $c_{1}$, and taking the anti-derivative again, we find that

$$
f(x)=x^{3}+x^{4}+c_{1} x+c_{2}
$$

for some constant $c_{2}$. In particular, one function that satisfies this condition is

$$
x^{3}+x^{4}
$$

(b) Similar to before we can take the anti-derivative to find that

$$
f^{\prime}(t)=2 e^{t}-3 \cos t+c_{1}
$$

for some constant $c_{1}$, and then

$$
f(t)=2 e^{t}-3 \sin t+c_{1} t+c_{2}
$$

for some constant $c_{2}$. Now plugging in $f(0)$ and $f(\pi)$ we get the equations

$$
2+c_{2}=0
$$

and

$$
2 e^{\pi}+c_{1} \pi+c_{2}=0
$$

Therefore, $c_{2}=-2$ and $c_{1}=\frac{2-2 e^{\pi}}{\pi}$. So the function is

$$
f(t)=2 e^{t}-3 \sin t+\frac{2-2 e^{\pi}}{\pi}-2
$$

(c) This one is slightly more tricky. Let's follow the hint. We calculate that

$$
g^{\prime}(x)=e^{-x}\left(f^{\prime}(x)-f(x)\right)=0
$$

This means that $g(x)=c$ for some constant $c$. Therefore

$$
f(x)=c e^{x}
$$

Plugging in $f(0)=1$ gives us that

$$
f(x)=e^{x}
$$

(d) Again we follow the hint and calculate

$$
g^{\prime}(x)=e^{x}\left(f^{\prime}(x)+f(x)\right)
$$

and

$$
g^{\prime \prime}(x)=e^{x}\left(f^{\prime \prime}(x)+2 f^{\prime}(x)+f(x)\right)
$$

so using the relation $f^{\prime \prime}(x)=f(x)$ we get

$$
g^{\prime \prime}(x)=e^{x}\left(2 f^{\prime}(x)+2 f(x)\right)=2 g^{\prime}(x)
$$

Moving forward with the hint, we calculate that

$$
h^{\prime}(x)=e^{-2 x}\left(g^{\prime}(x)-2 g(x)\right)=0
$$

Therefore

$$
h(x)=c_{1}
$$

for some constant $c_{1}$, so

$$
g^{\prime}(x)=c_{1} e^{2 x}
$$

so by finding the anti-derivative we find that

$$
g(x)=c_{2}+\frac{c_{1}}{2} e^{2 x}
$$

for some constant $c_{2}$ and finally we can conclude that

$$
f(x)=c_{2} e^{-x}+\frac{c_{1}}{2} e^{x}
$$

plugging in $f(0)=2$ and $f^{\prime}(0)=0$ gives us that

$$
f(x)=e^{x}+e^{-x}
$$


\end{document}