\documentclass[10pt]{article}
\usepackage[utf8]{inputenc}
\usepackage[T1]{fontenc}
\usepackage{amsmath}
\usepackage{amsfonts}
\usepackage{amssymb}
\usepackage[version=4]{mhchem}
\usepackage{stmaryrd}

\begin{document}
\begin{enumerate}
  \item Prove the following identities\\
(a)
\end{enumerate}

$$
\frac{1+\tanh x}{1-\tanh x}=e^{2 x}
$$

(b)

$$
\cosh 2 x=\cosh ^{2} x+\sinh ^{2} x
$$

(c)

$$
(\cosh x+\sinh x)^{n}=\cosh n x+\sinh n x
$$

For any real number $n$.\\
2. Find the derivative of the function.\\
(a) $y=\tan ^{-1}(\sinh x)$\\
(b) $y=\tanh ^{-1}\left(x^{3}\right)$\\
(c) $y=\operatorname{sech}(\tanh x)$\\
3. Verify that $f(x)=\sqrt{x}-\frac{1}{3} x$ satisfies the hypotheses of Rolle's Theorem on the interval $[0,9]$. Find all numbers $c$ satisfying the conclusion of the theorem.\\
4. Explain why, if the graph of a polynomial function has three $x$-intercepts, then it must have at least two points at which its tangent line is horizontal. Is this true for any function having three $x$-intercepts?\\
5. Show that the equation $x^{4}+4 x+c=0$ has at most two solutions which are real numbers.\\
6. Verify that the function satisfies the hypotheses of the MVT on the given interval. Then, find all values of $c$ that satisfy the conclusion of the MVT.\\
(a) $f(x)=\frac{x}{x+2}$ on $[1,4]$\\
(b) $f(x)=e^{-2 x}$ on $[0,3]$\\
7. A number $a$ is called a fixed point of a function $f$ if $f(a)=a$. Prove that if $f$ satisfies the hypothesis of the MVT and $f^{\prime}(x) \neq 1$ for all $x$, then $f$ has at most one fixed point. Hint: Suppose there are at least two fixed points, call them a and $b, a \neq b$. Then, use the MVT.


\end{document}