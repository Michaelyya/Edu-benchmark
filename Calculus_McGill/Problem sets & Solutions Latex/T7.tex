\documentclass[10pt]{article}
\usepackage[utf8]{inputenc}
\usepackage[T1]{fontenc}
\usepackage{amsmath}
\usepackage{amsfonts}
\usepackage{amssymb}
\usepackage[version=4]{mhchem}
\usepackage{stmaryrd}

\begin{document}
\begin{enumerate}
  \item Find $\frac{d y}{d x}$ by using implicit differentiation\\
(a) $y^{2}+x^{2}=1$\\
(c) $2\left(x^{2}+y^{2}\right)^{2}=9\left(x^{2}-y^{2}\right)$\\
(d) $\sin x \cos y=\sin ^{2}(x+y)$. (Check out\\
(b) $\sin x+\cos y=x^{3}-3 y^{2}$ the graph of this equation)
  \item Use implicit differentiation to find an equation for the tangent line to the curve at the given point.\\
(a) $x^{2}+2 x y+4 y^{2}=12$ at $(2,1)$\\
(b) $x^{2}+y^{2}=\left(2 x^{2}+2 y^{2}-x\right)^{2}$ at $\left(0, \frac{1}{2}\right)$\\
(c) $y^{2}\left(y^{2}-4\right)=x^{2}\left(x^{2}-5\right)$ at $(0,-2)$
  \item Use implicit differentiation to prove that
\end{enumerate}

$$
\frac{d}{d x}\left(\sec ^{-1}(x)\right)=\frac{1}{x \sqrt{x^{2}-1}}
$$

Hint: let $y(x)=\sec ^{-1}(x)$, so that $x=\sec (y(x))$. To simplify the expression you get in the end, it might be helpful to rewrite things using trig identities.\\
4. In this problem, we'll evaluate the derivative of $f(x)=(\sin x)^{\ln x}$ in two 'different' ways.\\
(a) Use the fact that $\ln x$ is the inverse of $e^{x}$ to write $f(x)=\exp (\ln (f(x)))$ and then use the Chain Rule to evaluate $f^{\prime}(x)$.\\
(b) Take logarithms on both sides of $f(x)=(\sin x)^{\ln x}$ and then use implicit differentiation to evaluate $f^{\prime}(x)$. Note: this technique is known as logarithmic differentiation.\\
5. For the following functions $y(x)$. Write down the domain and range of the function $y$ and find the derivative of the function.\\
(a) $y=\sin ^{-1}(\sqrt{\sin x})$\\
(c) $y=\ln (\sec x+\tan x)$\\
(b) $y=\sqrt{\tan ^{-1}(x)}$\\
(d) $y=\ln \left(x e^{x^{2}}\right)$


\end{document}