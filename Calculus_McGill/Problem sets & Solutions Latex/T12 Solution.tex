\documentclass[10pt]{article}
\usepackage[utf8]{inputenc}
\usepackage[T1]{fontenc}
\usepackage{amsmath}
\usepackage{amsfonts}
\usepackage{amssymb}
\usepackage[version=4]{mhchem}
\usepackage{stmaryrd}

\begin{document}
\begin{enumerate}
  \item For the following functions $f, g$ and real number $a$ verify the formula
\end{enumerate}

$$
\lim _{x \rightarrow a} \frac{f(x)}{g(x)}=\lim _{x \rightarrow a} \frac{f^{\prime}(x)}{g^{\prime}(x)}
$$

without using l'Hospital's Rule!\\
(a) $f(x)=3 x-6, g(x)=7 x-14, a=2$\\
(b) $f(x)=3 x^{2}+5 x, g(x)=10 x, a=0$\\
(c) $f(x)=x^{3}+6 x^{2}+11 x+6, g(x)=x^{3}-4 x^{2}+x+6, a=-1$\\
(d) $f(x)=(x-a) p(x), g(x)=(x-a) q(x)$, where $p, q$ are differentiable functions with continuous derivatives, such that $q(a) \neq 0$.

\section*{Solution}
(a)

$$
\lim _{x \rightarrow 2} \frac{3(x-2)}{7(x-2)}=\lim _{x \rightarrow 2} \frac{3}{7}=\frac{3}{7}
$$

We also have $f^{\prime}(x)=3, g^{\prime}(x)=7$, therefore

$$
\lim _{x \rightarrow 2} \frac{f^{\prime}(x)}{g^{\prime}(x)}=\frac{3}{7}
$$

(b)

$$
\lim _{x \rightarrow 0} \frac{3 x^{2}+5 x}{10 x}=\lim _{x \rightarrow 0} \frac{3 x+5}{10}=\frac{1}{2}
$$

We also have $f^{\prime}(x)=6 x+5, g^{\prime}(x)=10$, therefore

$$
\lim _{x \rightarrow 0} \frac{f^{\prime}(x)}{g^{\prime}(x)}=\frac{1}{2}
$$

(c) For this one it helps to start by noting that $f(-1)=g(-1)=0$. Which means that both polynomials have a factor of $x+1$. Using polynomial long division we can check that

$$
f(x)=(x+1)(x+2)(x+3)
$$

and

$$
g(x)=(x+1)(x-2)(x-3)
$$

therefore

$$
\lim _{x \rightarrow-1} \frac{f(x)}{g(x)}=\lim _{x \rightarrow-1} \frac{(x+2)(x+3)}{(x-2)(x-3)}=\frac{1}{6}
$$

We also have $f^{\prime}(x)=3 x^{2}+12 x+11, g^{\prime}(x)=3 x^{2}-8 x+1$, therefore

$$
\lim _{x \rightarrow 0} \frac{f^{\prime}(x)}{g^{\prime}(x)}=\frac{3-12+11}{3+8+1}=\frac{1}{6}
$$

(d) It's clear that

$$
\lim _{x \rightarrow a} \frac{f(x)}{g(x)}=\lim _{x \rightarrow a} \frac{p(x)}{q(x)}=\frac{p(a)}{q(a)}
$$

We also have $f^{\prime}(x)=(x-a) p^{\prime}(x)+p(x), g^{\prime}(x)=(x-a) q^{\prime}(x)+q(x)$, therefore

$$
\lim _{x \rightarrow a} \frac{f^{\prime}(x)}{g^{\prime}(x)}=\frac{(a-a) p^{\prime}(a)+p(a)}{(a-a) q^{\prime}(a)+q(a)}=\frac{p(a)}{q(a)}
$$

\begin{enumerate}
  \setcounter{enumi}{1}
  \item Evaluate the following limits using l'Hospital's Rule, if it applies.\\
(a) $\lim _{x \rightarrow \pi / 4} \frac{\sin x-\cos x}{\tan x-1}$\\
(d) $\lim _{x \rightarrow 0^{+}}(\tan (2 x))^{x}$\\
(b) $\lim _{x \rightarrow \infty} \frac{\ln \ln x}{x}$\\
(e) $\lim _{\theta \rightarrow \pi / 2} \frac{1-\sin \theta}{\csc \theta}$\\
(c) $\lim _{x \rightarrow 0^{+}}(\sin x)(\ln x)$\\
(f) $\lim _{x \rightarrow \infty}\left(1+\frac{a}{x}\right)^{b x}$
\end{enumerate}

\section*{Solution}
(a) We can apply l'Hospital's Rule, because when we evaluate at $\pi / 4$ we get $\frac{0}{0}$. Therefore

$$
\lim _{x \rightarrow \pi / 4} \frac{\sin x-\cos x}{\tan x-1}=\lim _{x \rightarrow \pi / 4} \frac{\cos x+\sin x}{\sec ^{2} x}=\frac{\cos (\pi / 4)+\sin (\pi / 4)}{\sec ^{2}(\pi / 4)}=\frac{\sqrt{2}}{2}
$$

Note that for this one, we do not need l'Hospital's Rule if we observe that

$$
\frac{\sin x-\cos x}{\tan x-1}=\cos x
$$

(b) We can apply l'Hospital's Rule as this evaluates to $\frac{\infty}{\infty}$. So we have

$$
\lim _{x \rightarrow \infty} \frac{\ln \ln x}{x}=\lim _{x \rightarrow \infty} \frac{1}{x} \cdot \frac{1}{\ln x}=0
$$

(c) There are several ways to do this one, but the intuition should be that $\sin x$ is roughly like $x$, when $x$ is close to 0 , and we know that $x$ is much faster than $\ln x$. So we want to compare both $\sin x$ and $\ln x$ to $x$. So we can write

$$
(\sin x)(\ln x)=\frac{\sin x}{x} \cdot x \ln x
$$

We can evaluate the limit on each of the two terms in the product above using a simple application of l'Hospital's Rule

$$
\begin{aligned}
& \lim _{x \rightarrow 0^{+}} \frac{\sin x}{x}=1 \\
& \lim _{x \rightarrow 0^{+}} x \ln x=0
\end{aligned}
$$

Therefore

$$
\lim _{x \rightarrow 0^{+}}(\sin x)(\ln x)=0
$$

(d) To apply l'Hospital's rule we can re-write at

$$
\lim _{x \rightarrow 0^{+}} e^{x \ln (\tan (2 x))}
$$

So we want to calculate the limit

$$
\lim _{x \rightarrow 0^{+}} x \ln (\tan (2 x))=\lim _{x \rightarrow 0^{+}} \frac{\ln (\tan (2 x))}{1 / x}
$$

this last limit we can apply l'Hospital's Rule to and is equal to

$$
\lim _{x \rightarrow 0^{+}}-2 x^{2} \frac{\sin (2 x)}{\cos (2 x)}=0
$$

So plugging this back in we see that

$$
\lim _{x \rightarrow 0^{+}}(\tan (2 x))^{x}=1
$$

(e) This limit does not satisfy the hypothesis for l'Hospital's rule. In fact, we can just plug in $\theta=\pi / 2$, so we get the limit is equal too 0 .\\
(f) For this one, we repeat the same trick of taking $\ln$. We rewrite it as

$$
\lim _{x \rightarrow \infty} e^{b x \ln \left(1+\frac{a}{x}\right)}
$$

So we want to find the limit

$$
\lim _{x \rightarrow \infty} b x \ln \left(1+\frac{a}{x}\right)=\lim _{x \rightarrow \infty} b \frac{\ln \left(1+\frac{a}{x}\right)}{1 / x}
$$

This final limit we can apply l'Hospital's rule to to get it's equal to

$$
\lim _{x \rightarrow \infty} \frac{a b}{1+\frac{a}{x}}=a b
$$

So plugging back in we can conclude that

$$
\lim _{x \rightarrow \infty}\left(1+\frac{a}{x}\right)^{b x}=e^{a b}
$$

\begin{enumerate}
  \setcounter{enumi}{2}
  \item Show that l'Hospital's Rule fails to yield a solution for $\lim _{x \rightarrow \infty} \frac{x}{\sqrt{x^{2}+1}}$. Evaluate the limit by other means.
\end{enumerate}

\section*{Solution}
We can apply l'Hospital's Rule to this limit to get

$$
\lim _{x \rightarrow \infty} \frac{\sqrt{x^{2}+1}}{x}
$$

which is the same limit, except we've switched the numerator and denominator. If we apply l'Hospital's Rule once again we will get our original limit back. So we have not made any progress. Instead we can divide the numerator and denominator by $x$ to get that

$$
\frac{x}{\sqrt{x^{2}+1}}=\frac{1}{\sqrt{1+1 / x^{2}}}
$$

and as $x$ goes to $\infty$, the fraction $\frac{1}{x^{2}}$ tends to 0 , therefore the whole limit tends to 1 .\\
4. For the following functions

\begin{itemize}
  \item Find the intervals on which $f$ is increasing or decreasing,
  \item Find the local maximum and minimum values of $f$,
  \item Find the interval of concavity and the inflection points,
  \item Use this information to sketch a graph of $f$.\\
(a) $f(x)=\frac{e^{x}}{x^{2}}$\\
(c) $f(x)=\frac{\ln x}{x^{2}}$\\
(b) $f(x)=\frac{1}{1+e^{-x}}$\\
(d) $f(x)=\frac{1}{x}+\ln x$
\end{itemize}

\section*{Solution}
(a) The domain of $f$ is the non-zero real numbers. There are two vertical asymptotes going to $+\infty$ as $x$ approaches 0 from both sides, and a horizontal asymptote going to 0 as $x$ goes to $-\infty$. This function has no $x$ - intercepts. We differentiate to find

$$
f^{\prime}(x)=\frac{x(x-2) e^{x}}{x^{4}}
$$

so its sign is determined by $x(x-2)$ which is negative for $0<x<2$ and positive elsewhere. Therefore the function is increasing for $x<0$ and decreasing for $0<x<2$ and increasing for $x>2$ and so it has a local minimum at $x=2$. We differentiate again to get

$$
f^{\prime}(x)=\frac{x^{4} e^{x}\left(x^{2}-4 x+8\right)}{x^{8}}
$$

Notice that this function is always positive. We can see that $x^{2}-4 x+8$ is always positive by seeing that it has no real roots. Therefore, the graph of $f$ is always concave up.\\
(b) The domain of $f$ is all the real numbers. It never has an $x$ intercept, and it has two horizontal asymptotes, tending to 1 as $x$ tends to $+\infty$ and tending to 0 as $x$ tends to $-\infty$, and also note that $f(0)=\frac{1}{2}$. Differentiating we get

$$
f^{\prime}(x)=\frac{e^{-x}}{1+e^{-x}}
$$

which is never 0 and is always positive, so this function is always increasing. Differentiating again we find

$$
f^{\prime \prime}(x)=\frac{e^{-x}\left(\left(e^{-x}\right)^{2}-1\right)}{\left(1+e^{-x}\right)^{4}}
$$

which is positive exactly when $e^{-x}>1$. Therefore $f^{\prime \prime}(x)$ is positive for $x>0$ and negative for $x<0$ and so it has an inflection point at $x=0$ and the graph is concave down for $x<0$ and concave up for $x>0$.\\
From sketching the function it might seem that if we shift the function down by $\frac{1}{2}$ then we get a symmetric odd function. This we can verify explicitly because $f(x)-\frac{1}{2}=\frac{1}{2} \frac{1-e^{-x}}{1+e^{-x}}$ and noting that

$$
\frac{1-e^{x}}{1+e^{x}}=-\frac{1-e^{-x}}{1+e^{x}}
$$

(c) The domain of $f$ is on the positive real numbers. It crosses the $x$-axis when $x=1$, and has a vertical asymptote at $x=0$, which goes to $-\infty$. Differentiating

$$
f^{\prime}(x)=\frac{1-2 \ln x}{x^{3}}
$$

Therefore, $f^{\prime}(x)$ is positive only for $x<e^{1 / 2}$, so the function is increasing for $x<e^{1 / 2}$ and has a maximum at $x=e^{1 / 2}$ and then decreases, and it tends to 0 as $x$ tends to $\infty$. Differentiating once more,

$$
f^{\prime \prime}(x)=\frac{-(5-6 \ln x)}{x^{4}}
$$

Therefore, $f^{\prime \prime}(x)$ is positive for $x>e^{5 / 6}$ so it is concave down for $x<e^{5 / 6}$ and then hits an inflection point at $e^{5 / 6}$ and is concave up after that.\\
(d) The domain of $f$ is on the positive real numbers. It has a vertical asymptote at $x=0$, going off to $\infty$. Differentiating we have

$$
f^{\prime}(x)=\frac{x-1}{x^{2}}
$$

which is positive for $x>1$ and negative for $x<1$, so the function is decreasing for $x<1$ and then hits a minimum at $x=1$ and then starts increasing. Differentiating again we have

$$
f^{\prime}(x)=\frac{2-x}{x^{3}}
$$

so the function is concave up for $x<2$ and concave down for $x>2$ and has an inflection point at $x=2$.


\end{document}