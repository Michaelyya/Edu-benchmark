\documentclass[10pt]{article}
\usepackage[utf8]{inputenc}
\usepackage[T1]{fontenc}
\usepackage{amsmath}
\usepackage{amsfonts}
\usepackage{amssymb}
\usepackage[version=4]{mhchem}
\usepackage{stmaryrd}

\begin{document}
\begin{enumerate}
  \item Sketch the graph of a function $f$ that satisfies all of the following conditions\\
(a) $\lim _{x \rightarrow 0} f(x)=+\infty$\\
(d) $\lim _{x \rightarrow 1} f(x)=0$\\
(b) $\lim _{x \rightarrow 2^{+}} f(x)=-\infty$\\
(e) $\lim _{x \rightarrow-1} f(x)$ does not exist.\\
(c) $\lim _{x \rightarrow 2^{-}} f(x)=3$
  \item Investigate the following limits. Back up your answers by sketching (or looking up) the graphs of the associated functions, and by plugging in individual values.\\
(a) $\lim _{x \rightarrow-1} x^{2}+1$\\
(e) $\lim _{x \rightarrow 1^{+}} \frac{x+1}{x^{3}-1}$\\
(b) $\lim _{x \rightarrow-1} \frac{x+1}{x^{3}-1}$\\
(c) $\lim _{x \rightarrow-1} \frac{x+1}{x^{3}+1}$\\
(f) $\lim _{x \rightarrow 1^{-}} \frac{x+1}{x^{3}-1}$\\
(d) $\lim _{x \rightarrow 0} \sin \left(\frac{1}{x}\right)$\\
(g) $\lim _{x \rightarrow 2} \frac{2-x}{\sqrt{x+2}-2}$
  \item Given that
\end{enumerate}

$$
\lim _{x \rightarrow 2} f(x)=4, \lim _{x \rightarrow 2} g(x)=-2, \lim _{x \rightarrow 2} h(x)=0
$$

find the value of the following limits if they exist, or explain why the limits don't exist.\\
(a) $\lim _{x \rightarrow 2}(f(x)+5 g(x))$\\
(c) $\lim _{x \rightarrow 2} \frac{f(x) g(x)}{h(x)}$\\
(b) $\lim _{x \rightarrow 2} g(x)^{3}$\\
(d) $\lim _{x \rightarrow 2} \cos (h(x))$\\
4. If the limit $\lim _{x \rightarrow a} f(x)$ does not exist and the limit $\lim _{x \rightarrow a} g(x)$ does not exist, does it follow that the limit

$$
\lim _{x \rightarrow a}(f(x)+g(x))
$$

does not exist as well?\\
5. In this exercise we will be using the $\epsilon-\delta$ definition of limits to calculate some limits.\\
(a) For $f(x)=x+1$, for each value of $\epsilon$ find a value of $\delta$ such that

$$
\text { if }|x-1|<\delta \text { then }|f(x)-f(1)|<\epsilon
$$

i. $\epsilon=0.1$\\
ii. $\epsilon=0.01$\\
iii. $\epsilon=0.001$

Can you write a formula for $\delta$ in terms of $\epsilon$ that will work for any value of $\epsilon$ ? Write the limit statement for $f(x)$ that we are trying to justify.\\
(b) For $f(x)=\frac{x}{5}$, for each value of $\epsilon$ find a value of $\delta$ such that

$$
\text { if }|x-3|<\delta \text { then }|f(x)-f(3)|<\epsilon
$$

i. $\epsilon=0.1$\\
ii. $\epsilon=0.01$\\
iii. $\epsilon=0.001$

Can you write a formula for $\delta$ in terms of $\epsilon$ that will work for any value of $\epsilon$ ? Write the limit statement for $f(x)$ that we are trying to justify.\\
(c) For $f(x)=x^{3}$, for each value of $\epsilon$ find a value of $\delta$ such that

$$
\text { if }|x-0|<\delta \text { then }|f(x)-f(0)|<\epsilon
$$

i. $\epsilon=0.1$\\
ii. $\epsilon=0.01$\\
iii. $\epsilon=0.001$

Can you write a formula for $\delta$ in terms of $\epsilon$ that will work for any value of $\epsilon$ ? Write the limit statement for $f(x)$ that we are trying to justify.\\
(d) For $f(x)=\frac{1}{x^{2}}$, for each value of $M$ find a value of $\delta$ such that

$$
\text { if }|x-0|<\delta \text { then } f(x)>M
$$

i. $M=10$\\
ii. $M=100$\\
iii. $M=1000$

Can you write a formula for $\delta$ in terms of $M$ that will work for any value of $M$ ? Write the limit statement for $f(x)$ that we are trying to justify.


\end{document}