\documentclass{article}

\begin{document}

\title{Solutions, Problem Set \#11}
\author{8.044 Statistical Physics I Spring Term 2013 \\
        Massachusetts Institute of Technology \\
        Physics Department}
\date{}

\maketitle

\section*{Problem 1: Ripplons}

a) $/X_{32}\pi/X_{2}F/X_{4}C/X_{78} /X_{32}\pi/X_{2}F/X_{4}C/X_{79}/X_{6}B/X_{79} /X_{6}B/X_{78}(2\pi)^2 k-volume/point  =L_{x}L_{y} L_{y}oints/k-volume\pi xL_{p} D(k) =(2\pi)^2 A=(2\pi)^2 \#(\pi) =\pi k2(\pi)D(k)\pi=bk^{3/2} =\pi\pi b\pi^{4/3} D(k) \frac{d\#}{4A \pi 1/3}=3 AD(\pi) = = \pi m =\pi^{1/3}$ \\
b) $/X_{6}B /X_{6}B/X_{78}/X_{6}B/X_{79} /X_{44}/X_{28} \epsilon/X_{29}\epsilon$ \\
c) $U=Z_{11}\pi(<n> +11 )D(\pi)d\pi=2Z \pi\pi +1D(\pi)d\pi e\pi=k_{B}T\pi_{12} 0 0\pi CA=\pi@U_{1}(\pi=k_{T}^{2}\pi=k_{B}T_{B})e_{1A}=\pi + \pi^{1/3}d\pi@T\pi Z\pi (e\pi=k_{B}T\pi_{1})^2 2 A 0\pi 3\pi b^{4/3} Ak^{7/3}xB =(k_{T}^{4/3} (\pi^{3/3})Z_{1}x e^{3} B dx\pi b_{40}(e^{x}2/T_{4} =1)$ \\
d) There is no energy gap behavior because there is no energy gap. For any $k_{B}T$ there are always oscillators with $~!<k_{B}T$.

\section*{Problem 2: Two-Dimensional Metal}

a) $e^{ik_{x}(x+L)}=e^{ik_{x}x}e^{ik_{x}L}=e^{ik_{x}x} (2\pi)k_{x}=n_{x}n_{x}=L_{1};L_{2};\cdots$ The same holds of $k_{y}. 2\pi~k = (n_{x}x^{+} +n_{y}y^{+})ni=L_{1};L_{2};\cdots$ \\
b) $~D(k) =\frac{\pi L}{2\pi^{2} ~}$ for all $k$ \\
c) $/X_{6}B/X_{79} /X_{6}B/X_{78}/X_{32}\epsilon/X_{2}\gamma$ \\
d) $N= \frac{8L^{2}}{(2\pi)^{2}}\pi(\pi F)^{2} (\pi F)^{2}=1 1\pi^{2}\pi^{2}(N/A)^{2}1=2$ \\
e) $\pi FaD(\pi) =a\pi N =a\pi d\pi =\pi^{2} Z_{0}^{2}F \pi Z_{F}aU=a\pi^{2}d\pi=\pi^{3} 0^{3}F=2N\pi_{F}^{3}$ \\
f) Thermal agitation only disturbs a fraction $k_{B}T=\pi_{F}$ of the total number of electrons, and imparts to them an energy of the order of $k_{B}T$. Thus the total increase in energy from the $T = 0$ value is proportional to $T^{2}$ and the heat capacity will be linearly proportional to $T$. 

\section*{Problem 3: Donor Impurity States in a Semiconductor}

a) $<n>!e^{(\pi)=k_{B}T}=e^{()=k_{B}T}=e^{(\pi)=k_{B}T} \pi^{3/2}V 2m_{e}D_{states} (\pi) = (\pi^{2}\pi^{2} ~^{2})^{1/2} NC=$

\end{document}