```
\documentclass[12pt]{article}
\usepackage{amsmath}
\usepackage{amssymb}

\begin{document}

\title{MASSACHUSETTS INSTITUTE OF TECHNOLOGY \\
Physics Department \\
8.044 Statistical Physics I Spring Term 2013 \\
Excited State Helium, He$^*$ \\
An Example of Quantum Statistics in a Two Particle System}
\date{}
\maketitle

By definition He$^*$ has one electron in the lowest energy single particle spatial state, 1S, and one in the first excited single particle spatial state, 2S. As such its atomic configuration is given as $(1S)^1(2S)^1$.

Since the two electrons are in different single particle spatial states, they are not required by the Pauli Principle to have oppositely directed projections of their spins. Electrons are Fermions (with S=1/2). Thus the total wavefunction must be anti-symmetric with respect to the interchange of the two electrons. For this system one can factor the total wavefunction into a spatial part and a spin part.

\[
\Psi_{total} = \Psi_{spacetotal}\Psi_{spin}
\]

The purpose of this example is to show that the anti-symmetry can be carried either by the spatial part or the spin part of the wavefunction, and that the two different cases produce different effects on the energy of the state when the coulomb interaction between the electrons is taken into consideration.

The rules for addition of angular momentum require that the allowed values of the sum of two angular momenta, $S_T=S_1+S_2$, each of magnitude 1/2, are $S_T=0$ and $S_T=1$.

For the case of S=1/2 the eigenfunctions of the z component of S are $\phi_{1/2}$ and $\phi_{-1/2}$, 

\[
\hat{S}_z\phi_{1/2}=\frac{1}{2}\phi_{1/2}
\]
and
\[
\hat{S}_z\phi_{-1/2}=-\frac{1}{2}\phi_{-1/2}.
\]

Without attention to symmetry or anti-symmetry there are $2 \times 2 = 4$ states available to the two spins. There should be the same number of spin states after symmetry is taken into account. Those states are given on the next page. Note that $\hat{S}_z|T\rangle = \hat{S}_z(1) + \hat{S}_z(2)$

\[
| 0,0\rangle = A|_{\uparrow 1}\downarrow 2\rangle - |_{\downarrow 2}\uparrow 1\rangle
\]

For this state,

\[
\hat{S}_T^2 | T,0\rangle = 0 
\]
and
\[
\hat{S}_{z}| T,0\rangle = 0 
\]
also
\[
\hat{P}_{1,2}\left( \frac{1}{\sqrt{2}} \right) = -\frac{1}{\sqrt{2}}
\]

It is called a singlet state.

\[
|+,1\rangle = A|_{\uparrow 1}\uparrow 2\rangle
\]
and
\[
|+,0\rangle = A|_{\uparrow 1}\downarrow 2\rangle + |_{\downarrow 2}\uparrow 1\rangle
\]
finally
\[
|+,-1\rangle = A|_{\downarrow 1}\downarrow 2\rangle
\]

For these states

\[
\hat{S}_T^2|+,\pm1\rangle = 2|+,\pm1\rangle
\]
and
\[
\hat{S}_z|+,\pm1\rangle = \pm1|+,\pm1\rangle
\]
also
\[
\hat{P}_{1,2}|+,\pm1\rangle = |+,\pm1\rangle
\]

These three states are collectively called the triplet state.

...

\end{document}
```

This is a draft version of LaTeX conversion, note that the alignment, style, and other formatting may need to be adjusted to fit your specific needs. 

Also please note that some terms, equations, and symbols might need more context to be appropriately translated into LaTeX (for example, the parameter A and up and down arrow symbol in your original text). 

Please, review this sample and let me know if there are any specific styles, alignments or other LaTeX commands you would like to incorporate.