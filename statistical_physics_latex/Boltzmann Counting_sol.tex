\documentclass{article}
\usepackage{amsmath}

\begin{document}

\title{MASSACHUSETTS INSTITUTE OF TECHNOLOGY \\ Physics Department \\ 8.044 Statistical Physics I Spring Term 2013 \\ Solutions to Problem Set \#5}
\maketitle

\section*{Problem 1: Correct Boltzmann Counting}

\subsection*{a)}
\begin{align*}
3N &= 24\pi mE / V^N \\
3N / V^N &= [2\pi mkT]^{3N/2} \quad \text{using } E=(3/2)NkT \\
S(N,V,T) &= k \ln  \\
&=k \ln n_{VN[2\pi mkT]^{3N/2}} \\
&= Nk \ln V + (3/2)Nk \ln[2\pi mkT]
\end{align*}

Now let $N \rightarrow N$, $V \rightarrow V$, and $T \rightarrow T$. Then as a result:

\begin{align*}
S &\rightarrow  Nk \ln  + (3/2)Nk \ln[2\pi mkT] : \\
&= Nk \ln V \\
\end{align*}

So $S$ does not approach $S$ because of the failure in the first term.

\subsection*{b)} 
The pressure is the same on both sides of the partition, so:
\begin{align*}
P &= N_1kT/V_1= N_2kT/V_2 \\
N_1kT/N_2kT&=\nu V/[1-\nu]V \\
\end{align*}

We can solve this to put $N_1$ and $N_2$ in terms of $\nu$.
\begin{align*}
N_2&=\nu N_1/(1+\nu^{-1}) \\
N_1+N_2&=(N_1 + \nu N_1) / [1 + (1+\nu^{-1})] \\
N_2&=N-N_1=(N/\nu^{-1}-N_1)/ \nu^{-2} \\
\end{align*}
\\
\\
Since the mixing takes place isothermally (because for ideal gases there is no interaction between molecules), the T term in our expression for $S$ of each gas does not change. \
...
\end{document}