\documentclass[10pt]{article}
\usepackage[utf8]{inputenc}
\usepackage[T1]{fontenc}
\usepackage{amsmath}
\usepackage{amsfonts}
\usepackage{amssymb}
\usepackage[version=4]{mhchem}
\usepackage{stmaryrd}
\usepackage{bbold}

%New command to display footnote whose markers will always be hidden
\let\svthefootnote\thefootnote
\newcommand\blfootnotetext[1]{%
  \let\thefootnote\relax\footnote{#1}%
  \addtocounter{footnote}{-1}%
  \let\thefootnote\svthefootnote%
}

%Overriding the \footnotetext command to hide the marker if its value is `0`
\let\svfootnotetext\footnotetext
\renewcommand\footnotetext[2][?]{%
  \if\relax#1\relax%
    \ifnum\value{footnote}=0\blfootnotetext{#2}\else\svfootnotetext{#2}\fi%
  \else%
    \if?#1\ifnum\value{footnote}=0\blfootnotetext{#2}\else\svfootnotetext{#2}\fi%
    \else\svfootnotetext[#1]{#2}\fi%
  \fi
}

\begin{document}
\section*{Lecture hours 8-10}
Definition (Linear Transformations). We define a linear transformation $T: \mathbb{R}^{n} \rightarrow$ $\mathbb{R}^{m}$ as a function with two properties:

\begin{enumerate}
  \item $T(\vec{x}+\vec{y})=T(\vec{x})+T(\vec{y})$ for all $\vec{x}, \vec{y} \in \mathbb{R}^{n}$ (we say T preserves vector addition)
  \item $T(\vec{x})=c T(\vec{x})$ for all $\vec{x} \in \mathbb{R}^{n}$ and $c \in \mathbb{R}$ (we say T preserves scalar multiplication)
\end{enumerate}

Definition (Matrix Multiplication - the column view). Suppose now that $A$ is an $m \times n$ matrix (so $A$ has $m$ rows and $n$ columns, it might be a coefficient matrix for a system of m equations and n unknowns). Also, let $\vec{x}$ be a vector with $n$ entries. We define the vector $A \vec{x}$ as follows: If $\vec{c}_{k}$ is the kth column of $A$, then

$$
A \vec{x}=\left[\begin{array}{cccc}
\mid & \mid & & \mid \\
\vec{c}_{1} & \vec{c}_{2} & \ldots & \vec{c}_{3} \\
\mid & \mid & & \mid
\end{array}\right]\left[\begin{array}{c}
x_{1} \\
x_{2} \\
\vdots \\
x_{n}
\end{array}\right]=x_{1} \vec{c}_{1}+x_{2} \vec{c}_{2}+\cdots+x_{n} \vec{c}_{n}
$$

That is, the product of a matrix and a vector is a linear combination of the columns of the matrix.

Problem 16 (Basis for a subspace). Suppose that $V$ is the set of all vectors $\left(x_{1}, x_{2}, x_{3}, x_{4}, x_{5}\right)$ in $\mathbb{R}^{5}$ such that

\[
\left\{\begin{array}{l}
x_{1}+2 x_{2}-2 x_{3}+2 x_{4}-x_{5}=0  \tag{4.1}\\
x_{1}+2 x_{2}-x_{3}+3 x_{4}-2 x_{5}=0 \\
2 x_{1}+4 x_{2}-7 x_{3}+x_{4}+x_{5}=0
\end{array}\right.
\]

a) Explain why $V$ is a subspace of $\mathbb{R}^{5}$.\\
b) Find a basis for $V$.

Problem 17 (Linear Transformations). Suppose that $T: \mathbb{R}^{n} \rightarrow \mathbb{R}^{m}$ and $S: \mathbb{R}^{m} \rightarrow \mathbb{R}^{l}$ are two linear transformations. Define a transformation $F: \mathbb{R}^{n} \rightarrow \mathbb{R}^{l}$ by

$$
F(\vec{v})=S(T(\vec{v})) .
$$

a) Explain why $F$ is a linear transformation.

Now suppose that $n=3, m=2$, and $l=3$, and $T$ is induced by the matrix

$$
\left[\begin{array}{ccc}
1 & 0 & 2 \\
0 & -1 & 2
\end{array}\right],
$$

and $S$ is induced by the matrix

$$
\left[\begin{array}{cc}
-2 & 0 \\
0 & 1 \\
1 & 3
\end{array}\right] .
$$

b) Find $F(\vec{x})$ for any $\vec{x} \in \mathbb{R}^{3}$.

Hint : Write $F(\vec{x})$ in terms of $F\left(\vec{e}_{1}\right), F\left(\vec{e}_{2}\right)$ and $F\left(\vec{e}_{3}\right)$.

Problem 18 (Linear Transformations). Let $T: \mathbb{R}^{n} \rightarrow \mathbb{R}^{n}$ be a linear transformation. In this case, the domain and codomain are the same, so we can repeat $T$. The linear transformation obtained by repeating $T k$ times is denoted $T^{k}$.\\
a) Find a linear transformation $T: \mathbb{R}^{2} \rightarrow \mathbb{R}^{2}$ with $T$ not equal to the zero transformation ${ }^{1}$, but $T^{2}$ equal to the zero transformation.\\
b) Find a linear transformation $T: \mathbb{R}^{3} \rightarrow \mathbb{R}^{3}$ with neither $T$ nor $T^{2}$ equal to the zero transformation, but $T^{3}$ equal to the zero transformation.

Problem 19 (Geometrical Linear Transformations). In each part below, you are given a matrix $A$. Describe the effect of the linear transformation $T(\vec{x})=A \vec{x}$ in words.\\
a) $A=\left[\begin{array}{cc}\sqrt{2} / 2 & -\sqrt{2} / 2 \\ \sqrt{2} / 2 & \sqrt{2} / 2\end{array}\right]$.\\
b) $A=\left[\begin{array}{cc}\cos \theta & -\sin \theta \\ \sin \theta & \cos \theta\end{array}\right]$, for some $\theta \in \mathbb{R}$.\\
c) $A=\left[\begin{array}{ccc}\cos \theta & 0 & -\sin \theta \\ 0 & 1 & 0 \\ \sin \theta & 0 & \cos \theta\end{array}\right]$, for some $\theta \in \mathbb{R}$.

\footnotetext{${ }^{1}$ the zero transformation is the linear transformation that sends every vector to the zero vector in the codomain
}Problem 20 (Geometrical Linear Transformations). Find all possible values of $\lambda \in \mathbb{R}$ for which there is a non-zero vector $\vec{v} \in \mathbb{R}^{2}$ with

$$
\left[\begin{array}{cc}
1 & -1 \\
0 & 2
\end{array}\right] \vec{v}=\lambda \vec{v} .
$$

Let $A$ be a $2 \times 2$ matrix. What does the equation $A \vec{v}=\lambda \vec{v}$ mean geometrically? Can you think of a matrix $A$ for which there are no non-zero vectors with this property?


\end{document}