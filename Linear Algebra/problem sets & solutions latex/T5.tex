\documentclass[10pt]{article}
\usepackage[utf8]{inputenc}
\usepackage[T1]{fontenc}
\usepackage{amsmath}
\usepackage{amsfonts}
\usepackage{amssymb}
\usepackage[version=4]{mhchem}
\usepackage{stmaryrd}
\usepackage{bbold}

\begin{document}
\section*{Lecture hours 9-11}
This Tutorial is a review for the midterm.

\section*{Definitions}
Definition (Subspace - Span version). A subspace of $\mathbb{R}^{n}$ is a set of vectors in $\mathbb{R}^{n}$ that can be described as a span of vectors.

Definition (Subspace - Standard version). A subspace of $\mathbb{R}^{n}$ subset $V$ of $\mathbb{R}^{n}$ with the following properties:\\
(i) $V$ is a non-empty set.\\
(ii) If $\vec{u}$ is in $V, \vec{u}$ is also in $V$ for any scalar $k \in \mathbb{R}$ (We say $V$ is closed under scalar multiplication.)\\
(iii) If $\vec{u}$ and $\vec{w}$ are in $V$, their sum $\vec{u}+\vec{w}$ is also in $V$. (We say $V$ is closed under addition.)

Definition (Basis). The vectors $\vec{v}_{1}, \vec{v}_{2}, \ldots, \vec{v}_{m}$ are a basis of a subspace $V$ if they span $V$ and are linearly independent. In other words, a basis of a subspace $V$ is the minimal set of vectors needed to span all of $V$.

Definition (Dimension of a subspace). The dimension of the subspace $V$ is the number of vectors in a basis of $V$.

Definition (Linear Transformations). We define a linear transformation $T: \mathbb{R}^{n} \rightarrow$ $\mathbb{R}^{m}$ as a function with two properties:

\begin{enumerate}
  \item $T(\vec{x}+\vec{y})=T(\vec{x})+T(\vec{y})$ for all $\vec{x}, \vec{y} \in \mathbb{R}^{n}$ (we say T preserves vector addition)
  \item $T(\vec{x})=c T(\vec{x})$ for all $\vec{x} \in \mathbb{R}^{n}$ and $c \in \mathbb{R}$ (we say T preserves scalar multiplication)
\end{enumerate}

Problem 21 (Subspace). Suppose that $V$ is the set of all solutions of the homogeneous system

\[
\left\{\begin{array}{l}
x_{1}+2 x_{2}-2 x_{3}+2 x_{4}-x_{5}=0  \tag{5.1}\\
x_{1}+2 x_{2}-x_{3}+3 x_{4}-2 x_{5}=0 \\
2 x_{1}+4 x_{2}-7 x_{3}+x_{4}+x_{5}=0
\end{array}\right.
\]

Show that $V$ is a subspace of $\mathbb{R}^{5}$.

Problem 22 (Basis of a subspace). Let $U$ be a subspace of $\mathbb{R}^{5}$ defined by $U=$ $\left\{\left(x_{1}, x_{2}, x_{3}, x_{4}, x_{5}\right) \in \mathbb{R}^{5}: x_{1}=3 x_{2}, x_{3}=7 x_{4}\right\}$. Find a basis for $U$.

Problem 23 (Basis of a subspace). Suppose $\left\{\vec{v}_{1}, \vec{v}_{2}, \vec{v}_{3}, \vec{v}_{4}\right\}$ is a basis of $\mathbb{R}^{4}$. Prove that

$$
\left\{\vec{v}_{1}+\vec{v}_{2}, \vec{v}_{2}+\vec{v}_{3}, \vec{v}_{3}+\vec{v}_{4}, \vec{v}_{4}\right\}
$$

is also a basis of $\mathbb{R}^{4}$.

Problem 24 (Linear Transformations). Find $a, b \in \mathbb{R}$ such that $T: \mathbb{R}^{3} \rightarrow \mathbb{R}^{2}$ defined by

$$
T(x, y, z)=(2 x-4 y+3 z+b, 6 x+a x y z)
$$

is a linear transformation.

Problem 25 (Linear Transformations). Let $T: \mathbb{R}^{n} \rightarrow \mathbb{R}^{n}$ be a linear transformation and let $\overrightarrow{v_{1}}, \ldots, \overrightarrow{v_{k}}$ be vectors in $R^{n}$. True or false? If false, give a counter-example. If true, explain why.\\
a) If the vectors $T\left(\overrightarrow{v_{1}}\right), \ldots, T\left(\overrightarrow{v_{k}}\right)$ are linear independent, then $\overrightarrow{v_{1}}, \ldots, \overrightarrow{v_{k}}$ are also linear independent.\\
b) If the vectors $\overrightarrow{v_{1}}, \ldots, v_{k}$ are linear independent, then $T\left(\overrightarrow{v_{1}}\right), \ldots, T\left(\overrightarrow{v_{k}}\right)$ are also linear independent.


\end{document}