\documentclass[10pt]{article}
\usepackage[utf8]{inputenc}
\usepackage[T1]{fontenc}
\usepackage{amsmath}
\usepackage{amsfonts}
\usepackage{amssymb}
\usepackage[version=4]{mhchem}
\usepackage{stmaryrd}
\usepackage{bbold}

\begin{document}
\section*{Lecture hours 3-4}
\section*{Definitions}
Definition (Rank of a matrix). The rank of a matrix is the number of leading ones in the rref of that matrix.

Definition (Linear Combination). A linear combination of the vectors $\bar{v}_{1}, \bar{v}_{2}, \ldots, \bar{v}_{n}$ is an expression of the form $c_{1} \bar{v}_{1}+c_{2} \bar{v}_{2}+\cdots+c_{n} \bar{v}_{n}$ where $c_{1}, c_{2}$, through $c_{n}$ are real numbers. So it's just a sum of multiples of the vectors $\bar{v}_{1}, \bar{v}_{2}, \ldots, \bar{v}_{n}$.

Definition (Span). The span of the vectors $\bar{v}_{1}, \bar{v}_{2}, \ldots, \bar{v}_{n}$ is all possible linear combinations of these vectors, and it is denoted by $\operatorname{span}\left(\bar{v}_{1}, \bar{v}_{2}, \ldots, \bar{v}_{n}\right)$.

Definition (Homogeneous System). A homogeneous system of linear equations is a system in which each equation has no constant term.

Problem 6 (Rank of a coefficient matrix). Suppose you have a system of three linear equations for two unknowns.\\
a) What is the largest possible rank the coefficient matrix could have? What is the smallest possible rank?\\
b) If the system is consistent, what is the largest possible number of free variables in the solution? What is the smallest possible number?\\
c) What are the possibilities for the number of solutions?

Now suppose you have a different system, this time there are three linear equations for four unknowns.\\
d) What is the largest possible rank the coefficient matrix could have? What is the smallest possible rank?\\
e) If the system is consistent, what is the largest possible number of free variables in the solution? What is the smallest possible number?\\
f) What are the possibilities for the number of solutions?

Problem 7 (Linear systems with parameters). For the linear system

$$
\begin{aligned}
x-y+2 z & =4 \\
3 x-2 y+9 z & =14 \\
2 x-4 y+a z & =b
\end{aligned}
$$

find real numbers $a$ and $b$ such that:\\
a) The system has a unique solution.\\
b) The system has infinitely many solutions.\\
c) The system is inconsistent.

Problem 8 (Span 1). Is the vector

$$
\left[\begin{array}{l}
1 \\
0
\end{array}\right]
$$

in the span of the vectors

$$
\vec{u}_{1}=\left[\begin{array}{c}
1 \\
-2
\end{array}\right], \vec{u}_{2}=\left[\begin{array}{l}
3 \\
5
\end{array}\right] ?
$$

Are there vectors in $\mathbb{R}^{2}$ that are not in the span of $\vec{u}_{1}$ and $\vec{u}_{2}$ ? Explain why or why not.

Problem 9 (Span 2). Consider the three vectors in $\mathbb{R}^{3}$ :

$$
\vec{v}_{1}=\left[\begin{array}{l}
1 \\
1 \\
0
\end{array}\right], \vec{v}_{2}=\left[\begin{array}{l}
1 \\
t \\
0
\end{array}\right], \vec{v}_{3}=\left[\begin{array}{c}
-1 \\
-1 \\
s
\end{array}\right]
$$

where $s, t \in \mathbb{R}$. What are the values of $s$ and $t$ so that $\vec{v}_{1}, \vec{v}_{2}$, and $\vec{v}_{3}$ span:\\
a) A line.\\
b) A plane.\\
c) All of $\mathbb{R}^{3}$.

Problem 10 (Homogeneous systems). Suppose you have a homogeneous system of three equations for three unknowns $x, y$, and $z$. The coefficient matrix of this system has rank 3 . What is the solution? Why?

Problem 11. (Linear combinations) The vectors $\vec{x}$ and $\vec{y}$ are in the span of the vectors $\vec{w}_{1}$ and $\vec{w}_{2}$. The vector $\vec{z}$ is a linear combination of $\vec{x}$ and $\vec{y}$. Is $\vec{z}$ in the span of $\vec{w}_{1}$ and $\vec{w}_{2}$ ? Why or why not?


\end{document}