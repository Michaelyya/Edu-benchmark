\documentclass[10pt]{article}
\usepackage[utf8]{inputenc}
\usepackage[T1]{fontenc}
\usepackage{amsmath}
\usepackage{amsfonts}
\usepackage{amssymb}
\usepackage[version=4]{mhchem}
\usepackage{stmaryrd}
\usepackage{bbold}

\begin{document}
\section*{Lecture hours 5-7}
\section*{Definitions}
Definition (Linear relations). Consider the vectors $\vec{v}_{1}, \vec{v}_{2}, \ldots, \vec{v}_{r}$ in $\mathbb{R}^{n}$. An equation of the form $c_{1} \vec{v}_{1}+\cdots+c_{r} \vec{v}_{r}=\overrightarrow{0}$ is called a linear relation among the vectors $\vec{v}_{1}, \vec{v}_{2}, \ldots, \vec{v}_{r}$. If at least one of the $c_{i}$ is nonzero, then we call this a nontrivial linear relation among $\vec{v}_{1}, \vec{v}_{2}, \ldots, \vec{v}_{r}$.

Definition (Linear Independent vectors). We say vectors $\vec{v}_{1}, \vec{v}_{2}, \ldots, \vec{v}_{r}$ in $\mathbb{R}^{n}$ are linearly independent if and only if the only linear relation between them is the trivial one. In other words, $\vec{v}_{1}, \vec{v}_{2}, \ldots, \vec{v}_{r}$ in $\mathbb{R}^{n}$ are linearly independent if and only if the only way that $c_{1} \vec{v}_{1}+\cdots+c_{r} \vec{v}_{r}=\overrightarrow{0}$ is if all the $c_{i}$ are 0 .

Definition (Subspace). A subspace of $\mathbb{R}^{n}$ is a non-empty set of vectors in $\mathbb{R}^{n}$ that can be described as a span of vectors.

Here is an equivalent definition of subspace:

Definition (Subspace). A subspace of $\mathbb{R}^{n}$ is a non-empty subset $V$ of $\mathbb{R}^{n}$ with the following properties:\\
(i) If $\vec{u}$ is in $V, \vec{u}$ is also in $V$ for any scalar $k \in \mathbb{R}$ (We say $V$ is closed under scalar multiplication.)\\
(ii) If $\vec{u}$ and $\vec{w}$ are in $V$, their sum $\vec{u}+\vec{w}$ is also in $V$. (We say $V$ is closed under addition.)

Definition (Basis). The vectors $\vec{v}_{1}, \vec{v}_{2}, \ldots, \vec{v}_{m}$ are a basis of a subspace $V$ if they span $V$ and are linearly independent. In other words, a basis of a subspace $V$ is the minimal set of vectors needed to span all of $V$.

Definition (Dimension of a subspace). The dimension of the subspace $V$ is the number of vectors in a basis of $V$.

Problem 12 (Linear dependence). True or false? If false, give a counter-example. If true, explain why.\\
a) If $\vec{v}_{1}, \vec{v}_{2}, \vec{v}_{3}$ are linearly dependent vectors in $\mathbb{R}^{2}$, then $\vec{v}_{3}$ is in the span of $\vec{v}_{1}$ and $\vec{v}_{2}$.\\
b) Any collection of 4 vectors in $\mathbb{R}^{3}$ is linearly dependent.

Problem 13 (Subspaces). Give examples for:\\
a) A subset $V$ of $\mathbb{R}^{2}$ that is closed under scalar multiplication, but not closed under addition.\\
b) A subset $V$ of $\mathbb{R}^{2}$ that is closed under addition, but not closed under scalar multiplication.

Problem 14 (Subspaces of $\mathbb{R}^{n}$ ). Give an example of:\\
a) A subspace of $\mathbb{R}$.\\
b) A subset of $\mathbb{R}^{2}$ that is not a subspace of $\mathbb{R}^{2}$. Explain why it is not a subspace.\\
c) A subspace of $\mathbb{R}^{3}$ of dimension 2. Explain why it has dimension 2.\\
d) A subset of $\mathbb{R}^{3}$ that contains infinitely many vectors, but is not a subspace of $\mathbb{R}^{3}$. Explain why it is not a subspace.

Problem 15 (Basis and dimension). Find a basis for $\operatorname{span}\left(\vec{v}_{1}, \vec{v}_{2}, \vec{v}_{3}, \vec{v}_{4}\right)$ where

$$
\vec{v}_{1}=\left[\begin{array}{l}
1 \\
1 \\
2
\end{array}\right], \quad \vec{v}_{2}=\left[\begin{array}{c}
-1 \\
0 \\
3
\end{array}\right], \quad \vec{v}_{3}=\left[\begin{array}{l}
3 \\
2 \\
1
\end{array}\right], \quad \vec{v}_{4}=\left[\begin{array}{l}
0 \\
1 \\
5
\end{array}\right]
$$


\end{document}