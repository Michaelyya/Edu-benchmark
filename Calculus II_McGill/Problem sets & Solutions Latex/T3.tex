\documentclass[10pt]{article}
\usepackage[utf8]{inputenc}
\usepackage[T1]{fontenc}
\usepackage{amsmath}
\usepackage{amsfonts}
\usepackage{amssymb}
\usepackage[version=4]{mhchem}
\usepackage{stmaryrd}
\usepackage{bbold}

\begin{document}
\section*{Math 141 Tutorial 3}
\section*{Main problems}
\begin{enumerate}
  \item Suppose that $f, g:[a, b] \rightarrow \mathbb{R}$ are continuous functions and let $c \in \mathbb{R}$ be such that $a<c<b$. Given that
\end{enumerate}

$$
\int_{a}^{c} f(x) \mathrm{d} x=4, \quad \int_{a}^{b} f(x) \mathrm{d} x=-2, \quad \int_{a}^{c} g(x) \mathrm{d} x=-1, \quad \int_{c}^{b} g(x) \mathrm{d} x=3
$$

determine the value of each of the following integrals.\\
(a) $\int_{a}^{c}(f(x)+2 g(x)) \mathrm{d} x$\\
(b) $\int_{c}^{b} f(x) \mathrm{d} x$\\
(c) $\int_{a}^{b}(2 f(x)-5 g(x)) \mathrm{d} x$\\
2. Given that

$$
\int_{0}^{\pi} \sin (x) \mathrm{d} x=2 \quad \text { and } \quad \int_{0}^{\pi} \sin ^{2}(x) \mathrm{d} x=\frac{\pi}{2}
$$

and

$$
\int_{-\pi}^{0} \sin (x) \mathrm{d} x=-2 \quad \text { and } \quad \int_{-\pi}^{0} \sin ^{2}(x) \mathrm{d} x=\frac{\pi}{2}
$$

determine the value of each of the following integrals.\\
(a) $\int_{0}^{\pi}\left(2 \sin ^{2}(x)-\pi \sin (x)\right) \mathrm{d} x$\\
(b) $\int_{0}^{\pi} \cos ^{2}(x) \mathrm{d} x$\\
(c) $\int_{-\pi}^{\pi} \sin (x)(\sin (x)+1) d x$\\
3. For each function $f$ below, find constants $m$ and $M$ such that

$$
m \leq \int_{a}^{b} f(x) d x \leq M
$$

(a) $f(x)=x^{3}+1$ with $a=0$ and $b=2$.\\
(b) $f(x)=\ln \left(x^{2}+4 x+14\right)$ with $a=-4$ and $b=2$.\\
4. Consider the function $F$ given by

$$
F(x)=\int_{0}^{x} \cos (t) e^{t^{2}} \mathrm{~d} t
$$

Find where the local maxima and minima of $F(x)$ on $(0,2 \pi)$ occur. (Do not try and evaluate $F$ at these points!)

\section*{Challenge Problems}
\begin{enumerate}
  \setcounter{enumi}{4}
  \item Compute the following limits\\
(a) $\lim _{n \rightarrow \infty} \frac{1}{n}\left(\sqrt{\frac{1}{n}}+\sqrt{\frac{2}{n}}+\cdots+\sqrt{\frac{n}{n}}\right)$\\
(b) $\lim _{n \rightarrow \infty} \sum_{i=1}^{n} \frac{2 \pi}{n} \sin \left(\frac{2 \pi i}{n}\right)$
\end{enumerate}

Hint: how do these limits relate to Riemann sums?\\
6. Let

$$
g(x)=\int_{0}^{h(x)} \frac{1}{\sqrt{1+t^{4}}} \mathrm{~d} t, \quad h(x)=\int_{0}^{\cos (x)}\left(1+\sin \left(s^{2}\right)\right) \mathrm{d} s
$$

What's the value of $g^{\prime}\left(\frac{\pi}{2}\right)$ ?


\end{document}