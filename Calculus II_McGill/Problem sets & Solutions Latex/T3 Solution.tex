\documentclass[10pt]{article}
\usepackage[utf8]{inputenc}
\usepackage[T1]{fontenc}
\usepackage{amsmath}
\usepackage{amsfonts}
\usepackage{amssymb}
\usepackage[version=4]{mhchem}
\usepackage{stmaryrd}
\usepackage{bbold}

\begin{document}
\section*{Math 141 Tutorial 3 Solutions}
\section*{Main problems}
\begin{enumerate}
  \item Suppose that $f, g:[a, b] \rightarrow \mathbb{R}$ are continuous functions and let $c \in \mathbb{R}$ be such that $a<c<b$. Given that
\end{enumerate}

$$
\int_{a}^{c} f(x) \mathrm{d} x=4, \quad \int_{a}^{b} f(x) \mathrm{d} x=-2, \quad \int_{a}^{c} g(x) \mathrm{d} x=-1, \quad \int_{c}^{b} g(x) \mathrm{d} x=3
$$

determine the value of each of the following integrals.\\
(a) $\int_{a}^{c}(f(x)+2 g(x)) \mathrm{d} x$\\
(b) $\int_{c}^{b} f(x) \mathrm{d} x$\\
(c) $\int_{a}^{b}(2 f(x)-5 g(x)) \mathrm{d} x$

Solution:\\
Note that $f$ and $g$ are continuous on $[a, b]$ and thus continuous on $[a, c]$ and $[c, b]$. This ensures that $f$ and $g$ are Riemann integrable on these intervals and guarantees the existence of their respective integrals on $[a, c]$ and $[c, b]$.\\
(a) Using the properties shown in the last problem, we have

$$
\begin{aligned}
\int_{a}^{c}(f(x)+2 g(x)) \mathrm{d} x & =\int_{a}^{c} f(x) \mathrm{d} x+\int_{a}^{c} 2 g(x) \mathrm{d} x \\
& =\int_{a}^{c} f(x) \mathrm{d} x+2 \int_{a}^{c} g(x) \mathrm{d} x \\
& =4+2 \cdot(-1) \\
& =2 .
\end{aligned}
$$

(b) By additivity, we have

$$
\int_{a}^{b} f(x) \mathrm{d} x=\int_{a}^{c} f(x) \mathrm{d} x+\int_{c}^{b} f(x) \mathrm{d} x .
$$

Or, equivalently,

$$
\int_{c}^{b} f(x) \mathrm{d} x=\int_{a}^{b} f(x) \mathrm{d} x-\int_{a}^{c} f(x) \mathrm{d} x=-2-4=-6 .
$$

(c) First, we note that

$$
\int_{a}^{b} g(x) \mathrm{d} x=\int_{a}^{c} g(x) \mathrm{d} x+\int_{c}^{b} g(x) \mathrm{d} x=-1+3=2 .
$$

Hence, by the properties shown in the last question,

$$
\begin{aligned}
\int_{a}^{b}(2 f(x)-5 g(x)) \mathrm{d} x & =\int_{a}^{b} 2 f(x) \mathrm{d} x+\int_{a}^{b}(-5 g(x)) \mathrm{d} x \\
& =2 \int_{a}^{b} f(x) \mathrm{d} x-5 \int_{a}^{b} g(x) \mathrm{d} x \\
& =2(-2)-5(2) \\
& =-14
\end{aligned}
$$

\begin{enumerate}
  \setcounter{enumi}{1}
  \item Given that
\end{enumerate}

$$
\int_{0}^{\pi} \sin (x) \mathrm{d} x=2 \quad \text { and } \quad \int_{0}^{\pi} \sin ^{2}(x) \mathrm{d} x=\frac{\pi}{2}
$$

and

$$
\int_{-\pi}^{0} \sin (x) \mathrm{d} x=-2 \quad \text { and } \quad \int_{-\pi}^{0} \sin ^{2}(x) \mathrm{d} x=\frac{\pi}{2}
$$

determine the value of each of the following integrals.\\
(a) $\int_{0}^{\pi}\left(2 \sin ^{2}(x)-\pi \sin (x)\right) \mathrm{d} x$\\
(b) $\int_{0}^{\pi} \cos ^{2}(x) \mathrm{d} x$\\
(c) $\int_{-\pi}^{\pi} \sin (x)(\sin (x)+1) \mathrm{d} x$

Solution:

Note that all functions here are continuous on $\mathbb{R}$ and therefore Riemann integrable on any interval of the form $[a, b] \subset \mathbb{R}$.\\
(a) We have

$$
\begin{aligned}
\int_{0}^{\pi}\left(2 \sin ^{2}(x)-\pi \sin (x)\right) \mathrm{d} x & =2 \int_{0}^{\pi} \sin ^{2}(x) \mathrm{d} x-\pi \int_{0}^{\pi} \sin (x) \mathrm{d} x \\
& =2\left(\frac{\pi}{2}\right)-\pi(2) \\
& =-\pi
\end{aligned}
$$

(b) Recall that $\cos ^{2}(x)=1-\sin ^{2}(x)$ for all $x \in \mathbb{R}$. Thus,

$$
\begin{aligned}
\int_{0}^{\pi} \cos ^{2}(x) \mathrm{d} x=\int_{0}^{\pi}\left(1-\sin ^{2}(x)\right) \mathrm{d} x & =\int_{0}^{\pi} \mathrm{d} x-\int_{0}^{\pi} \sin ^{2}(x) \mathrm{d} x \\
& =\pi-\frac{\pi}{2} \\
& =\frac{\pi}{2}
\end{aligned}
$$

Page 2\\
(c) Linearity of the integral tells us that

$$
\begin{aligned}
\int_{-\pi}^{\pi} \sin (x)(\sin (x)+1) \mathrm{d} x & =\int_{-\pi}^{\pi}\left(\sin ^{2}(x)+\sin (x)\right) \mathrm{d} x \\
& =\int_{-\pi}^{\pi} \sin ^{2}(x) \mathrm{d} x+\int_{-\pi}^{\pi} \sin (x) \mathrm{d} x \\
& =\int_{-\pi}^{0} \sin ^{2}(x) \mathrm{d} x+\int_{0}^{\pi} \sin ^{2}(x) \mathrm{d} x+\int_{-\pi}^{0} \sin (x) \mathrm{d} x+\int_{0}^{\pi} \sin (x) \mathrm{d} x \\
& =\frac{\pi}{2}+\frac{\pi}{2}-2+2 \\
& =\pi
\end{aligned}
$$

\begin{enumerate}
  \setcounter{enumi}{2}
  \item Using the comparison principle, for each function $f$ below find constants $m$ and $M$ such that
\end{enumerate}

$$
m \leq \int_{a}^{b} f(x) d x \leq M
$$

(a) $f(x)=x^{3}+1$ with $a=0$ and $b=2$.\\
(b) $f(x)=\ln \left(x^{2}+4 x+14\right)$ with $a=-4$ and $b=2$.

Solution:

For each question, we'll find constants $A$ and $B$ such that $A<=f(x)<=B$ for all $x \in[a, b]$, then using the comparison principle combined with the linearity properties of the integral to get $m=A(b-a)$ and $M=B(b-a)$ :

$$
A(b-a)=A \int_{a}^{b} 1 d x \leq \int_{a}^{b} f(x) \leq B \int_{a}^{b} 1 d x=B(b-a) .
$$

(a) We start by trying to find the local minimum and maximum of $f(x)=x^{3}+1$ on $[0,2]$. Differentiating $f(x)$, we get

$$
f^{\prime}(x)=3 x^{2}
$$

and solving for $x$ such that $f^{\prime}(x)=0$ gives $f^{\prime}(x)=3 x^{2}=0$ at $x=0$. We can't conclude whether $x=0$ is a local min/max just yet because

$$
f^{\prime \prime}(0)=6(0)=0,
$$

i.e., $x=0$ is an inflection point. Next, we'll check $f(x)$ at the endpoints of $[0,2]$ :

$$
\begin{aligned}
& f(0)=0^{3}+1=1 \\
& f(2)=2^{3}+1=9 .
\end{aligned}
$$

Since we know $f(x)$ is an increasing function on $[0,2]\left(f^{\prime}(x) \geq 0\right.$ for all $\left.x \in[0,2]\right)$, we can conclude that

$$
1 \leq f(x) \leq 9,
$$

\section*{Page 3}
for all $x \in[0,2]$. Hence by the work above, we get

$$
2=1 \cdot 2 \leq \int_{0}^{2} 3 x^{2}+1 d x \leq 9 \cdot 2=18
$$

That is, $m=2$ and $M=18$.\\
(b) We approach this problem similarly to the previous one. Differentiating $f(x)$, we get

$$
f^{\prime}(x)=\frac{2 x+4}{x^{2}+4 x+14}
$$

Solving for $f^{\prime}(x)=0$, we get $x=-2$. Next, we compute $f(-4), f(-2)$, and $f(2)$ to find the min/max of $f(x)$ on $[-4,2]$. In doing so, we see that

$$
f(-4) \approx 2.6391 \quad f(-2) \approx 2.3026 \quad f(2) \approx 3.2581
$$

Hence by rounding down/up, we have $2.3 \leq f(x) \leq 3.3$ and so

$$
13.8=\leq 2.3 \int_{-4}^{2} 1 d x \leq \int_{-4}^{2} \ln \left(x^{2}+4 x+14\right) d x \leq 3.3 \int_{-4}^{2} 1 d x=19.8
$$

i.e., we could have $m=13.8$ and $M=19.8$.\\
4. Consider the function $F$ given by

$$
F(x)=\int_{0}^{x} \cos (t) e^{t^{2}} \mathrm{~d} t
$$

Find where the local maxima and minima of $F(x)$ on $(0,2 \pi)$ occur. (Do not try and evaluate $F$ at these points!)\\
Solution:\\
By the fundamental theorem of calculus, $F$ is differentiable with derivative

$$
F^{\prime}(x)=\cos (x) e^{x^{2}}
$$

Recalling techniques from Calculus 1 , we begin by searching for $x \in(0,2 \pi)$ such that $F^{\prime}(x)=0$. Since we know what $F^{\prime}(x)$ is, this is the same as looking for $x \in(0,2 \pi)$ such that

$$
\cos (x) e^{x^{2}}=0
$$

Since $e^{x^{2}}>0$ for all $x \in \mathbb{R}$, the above can only occur when $\cos (x)=0$. Since we are restricting ourselves to $x \in(0,2 \pi)$, this leaves only the following possibilities:

$$
x=\frac{\pi}{2} \quad \text { or } \quad x=\frac{3 \pi}{2} .
$$

Using the second derivative test with $F^{\prime \prime}(x)=2 x e^{x^{2}} \cos (x)-\sin (x) e^{x^{2}}$, we have

$$
F^{\prime \prime}\left(\frac{\pi}{2}\right)<0 \quad \text { and } \quad F^{\prime \prime}\left(\frac{3 \pi}{2}\right)>0
$$

The relative min is thus at $\frac{3 \pi}{2}$ and the relative max is at $\frac{\pi}{2}$.

\section*{Challenge Problems}
\begin{enumerate}
  \setcounter{enumi}{4}
  \item Compute the following limits\\
(a) $\lim _{n \rightarrow \infty} \frac{1}{n}\left(\sqrt{\frac{1}{n}}+\sqrt{\frac{2}{n}}+\cdots+\sqrt{\frac{n}{n}}\right)$\\
(b) $\lim _{n \rightarrow \infty} \sum_{i=1}^{n} \frac{2 \pi}{n} \sin \left(\frac{2 \pi i}{n}\right)$
\end{enumerate}

Hint: how do these limits relate to Riemann sums?\\
6. Let

$$
g(x)=\int_{0}^{h(x)} \frac{1}{\sqrt{1+t^{4}}} \mathrm{~d} t, \quad h(x)=\int_{0}^{\cos (x)}\left(1+\sin \left(s^{2}\right)\right) \mathrm{d} s
$$

What's the value of $g^{\prime}\left(\frac{\pi}{2}\right)$ ?


\end{document}