\documentclass{article}
\usepackage{amsmath}
\usepackage{amssymb}

\begin{document}
\title{Fall 2017 14.01 Problem Set 7}
\maketitle

\section*{Problem 1: True or False (20 points)}
Determine whether the following statements are True or False. Explain your answer.

\begin{enumerate}
\item Consider the Cournot duopoly model with linear demand and asymmetric costs, that is, the cost functions are $C(q_1) = c_1q_1$ for firm 1 and $C(q_2) = c_2q_2$ for firm 2 (where $c_1$ and $c_2$ are constants).
   \begin{enumerate}
   \item[(a)] (5 points) Claim 1: both firms will be producing strictly positive quantities.
   \item[(b)] (5 points) Claim 2: the firm with the lowest cost will be producing a higher quantity than its competitor. 
   \end{enumerate}
   
\item (5 points) In a competitive labor market, the market labor supply curve is always upward sloping.

\item (5 points) Consider a price-taking firm in the short run (capital is fixed). If the product produced by a firm becomes more valuable (i.e. its price increases), then the firm will respond by increasing its labor demand. 
\end{enumerate}

\section*{Problem 2: Labor Demand (30 points)}
Suppose that a representative, perfectly competitive, firm in the market has production function $F(K, L) = K^{\frac{1}{2}}L^{\frac{1}{2}}$ , the price of the firm’s product is equal to $p$, the price per unit of capital is $r$, and the cost per unit of labor is $w$.

\section*{Problem 3: Taxes and the Labor Market (20 points)}
Suppose a worker has preferences over consumption and leisure that can be represented by the following utility function: $U = \ln (c) + \ln (l)$ .

\section*{Problem 4: Firm, Labor Market Equilibrium (30 points)}
Suppose that there are two types of workers in the economy: domestic workers and immigrant workers. All workers have preferences over consumption $(c)$ and labor `$(\ell)$ that can be represented by the following utility function: $U = \ln (c) - \frac{\ell^{1+\epsilon}}{1 + \epsilon}$
\end{document}