\documentclass{article}
\usepackage{amsmath}

\begin{document}

\title{Fall 2018 14.01 Problem Set 9}
\maketitle{}

\section*{Problem 1: True or False (20 points)} 
Determine whether the following statements are True or False. Explain your answer.\\

\noindent
1. (5 points) Suppose the cost of making a car is cheapest in Japan. Then Japan should specialize in producing cars.\\
2. (5 points) In order for an IRA to encourage savings (relative to a regular savings account), the substitution effect of higher interest rates must dominate the income effect.\\
3. (5 points) An index fund is a portfolio of stocks that tracks a broader index such as the Dow Jones Industrial Average or the S\&P 500. Investing in an index fund is better than investing in the stock of an individual company because the index fund always has higher returns.\\
4. (5 points) Consider the effect of interest rates on consumption today. Increasing interest rates always has a negative substitution effect (decreases consumption today) and a positive income effect (increases consumption today).\\

\section*{Problem 2: Trade and Production Possibilities Frontier (20 points)} 
Consider the production of wine and cheese in France and Spain. This table gives the number of necessary hours to produce each (labor is the only input):\\

\noindent
\begin{tabular}{lcc}
 & France & Spain\\
1 Kilo of Cheese & 4 & 6\\
1 Bottle of wine & 6 & 12
\end{tabular}

\noindent
1. (5 points) For each good, which country has an absolute advantage? For each good, which country has a comparative advantage? \\
2. (5 points) Is it in Spain’s interest to develop trade relationships with France? Is it in France’s interest to trade with Spain? Is France more competitive at producing both goods? \\
3. (5 points) Suppose that France and Spain are under autarky (no trade). Draw the production possibility frontier for each country for the number of goods they can produce in one day (24 hours, one worker). \\
4. (5 points) France and Spain decide to trade and suppose they agree to trade one bottle of wine for k kilos of cheese. What values of k would make both France and Spain strictly better off under trade? Draw the new consumption set for each country under trade. \\
\end{document}
