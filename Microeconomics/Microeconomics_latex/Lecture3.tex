\documentclass{article}
\usepackage{amsmath}
\begin{document}

\section{International Trade}
\subsection{Lecture 18: International Trade}
\subsubsection{What is International Trade?}
\begin{itemize}
\item Autarky - an environment in which trade does not exist
\item trade deficit = imports - exports
\end{itemize}

\subsubsection{Comparative Advantage and Gains from Trade}
\begin{itemize}
\item We say a country has a comparative advantage in the production of a good when the opportunity cost of producing a particular good is lower in any one country.
\item Differences in opportunity costs lead to comparative advantage in different goods
\item Even when countries have an absolute advantage in producing a good, there can be comparative be a comparative advantage
\item When countries have different comparative advantages in production of different goods,...
\end{itemize}

\subsubsection{Welfare Implications from International Trade}
\begin{itemize}
\item In competitive model, opening to trade unambiguously increases total welfare but usually at the expense of either consumers or producers
\end{itemize}

\subsubsection{Trade Policy}
\begin{itemize}
\item Effects of import tariffs (tax levied on imported goods) and quotas
\end{itemize}

\subsubsection{TO KNOW- Conceptual Understanding}
...

\section{Uncertainty}
\subsection{Lecture 19 - Uncertainty}
\subsubsection{Expected Utility and Expected Value}
...

\subsubsection{Risk preferences}
...

\subsubsection{Applications}
...

\section{Capital Supply and Capital Markets}
\subsection{Lecture 20 - Capital Supply and Capital Markets}
\subsubsection{Capital and Intertemporal Choice}
...

\subsubsection{Intertemporal choice}
...

\subsubsection{Present Value}
...

\subsection{Lecture 21 - Capital Market}
...

\section{Equity and Efficiency}
\subsection{Lecture 22: Equity and Efficiency}
\subsubsection{Choosing the Socially Optimal Allocation}
...

\section{Taxation and Redistribution}
\subsection{Lecture 23: Taxation and Redistribution}
\end{document}