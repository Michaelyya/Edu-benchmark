\documentclass{article}
\usepackage{amsmath}
\usepackage{hyperref}

\title{Fall 2018 14.01 Problem Set 8}
\author{MIT OpenCourseWare}
\date{Fall 2018}

\begin{document}
\maketitle

\section*{Problem 1: True or False (20 points)}

Determine whether the following statements are True or False. Explain your answer.

\begin{enumerate}
\item (4 points) Raising the minimum wage may improve the welfare of workers, but will always lead to an increase in the deadweight loss.
\item (4 points) An increase in the interest rate has an ambiguous effect on the savings of a utility maximizing household.
\item (4 points) A student starting a 3 year long post-doc at MIT is considering two alternative car rental programs to use during those 3 years. Program 1 charges a \$1200 initial membership fee and then \$120 per year. Program 2 charges a \$360 initial membership fee and then \$240 per year. The student will always strictly prefer program 2. Assume $r > 0$.
\item (4 points) Allen is working, consuming, and saving rationally for retirement. A rise in real interest rates will definitely lead Allen to optimally decide to save more for retirement and consume less today.
\item (4 points) Suppose that interest rates are at 2 percent and a firm is considering a project with strictly positive net present value. If interest rates increase to 4 percent, the firm will still decide to make the investment to start that project.
\end{enumerate}

\section*{Problem 2: Monopsony and the labor market (30 points)}

Suppose that a logging company in Northern Carolina faces a perfectly competitive market for the lumber it produces (that is the company takes the price of lumber $p_1$ as given). However, the logging company is the only employer in the area and has a monopsony.

\begin{enumerate}
\item ((8 points) Suppose that workers in the area (employed by the logging company) are identical and have utility over consumption and labor given by $u(c, \ell) = c^{2/3} - 3\ell^{2/3}$ and earn $w$ for each unit of labor supplied. The price of one unit of consumption is equal to 1, and the only income a worker has is his labor income. Find the amount of labor that each worker will supply as a function of $w$.
\item (2 points) Suppose that there are 81 workers in the area. Find the market labor supply as a function of wage.
\item (8 points) Suppose that the firm’s production of lumber is given by $F(L) = 240L^{1/4}$ and the price of lumber is equal to $p = 9$ per unit. Recall that the firm has a monopsony over the labor market (they are the only employers in the area). Write the firm’s total revenue and total cost as a function of $w$.
\item (4 points) Find the wage that the firm sets and the amount of labor the firm demands.
\item (8 points) Suppose that the government wants to improve total surplus by imposing a minimum wage in North Carolina. What wage should the government set?
\end{enumerate}

\section*{Problem 3: Intertemporal Choice (20 points)}

A household has to decide how much to consume during their working age and how much to save for retirement. We will model this as if there were two periods: period 1 is the working age, while period 2 is retirement. Suppose that we can represent the preferences of this household with the utility function $U(c_1, c_2) = c_1^{1-\sigma}/(1-\sigma) + c_2^{1-\sigma}/(1-\sigma)$, where $c_1$ is consumption in period 1, and $c_2$ is consumption in period 2 and $\sigma > 0$.

Buying 1 unit of consumption costs \$1 in both periods. Income in period 1 is $W$ dollars and zero during retirement. The household can save at the market interest rate $r$. Assume that the household gets no utility from leaving any money behind after death.

\begin{enumerate}
\item (5 points) What is the price today of one unit of consumption during retirement? Why?
\item (5 points) Write an expression for the household’s budget constraint in terms of today’s value of consumption and income.
\item (5 points) How much of its income will the household consume and how much will it save given the interest rate $r$?
\item (5 points) First consider the case when $\sigma = 1/2$. Does increasing interest rates increase or decrease savings? Next consider the case when $\sigma = 2$. Does increasing interest rates increase or decrease savings in these two cases? If the qualitative effect of interest rates on savings are different in these two cases, provide some intuition for why this might happen.
\end{enumerate}

\section*{Problem 4: Equilibrium interest rate (30 points)}

Suppose that agents live for two periods $t = 1, 2$ and die. Agents have wealth in the first period $W_1 > 0$ but no wealth in the second period ($W_2 = 0$). Additionally agents have utility over consumption in the first and second period, given by $u(c_1, c_2) = c_1^{1-\sigma}/(1-\sigma) + c_2^{1-\sigma}/(1-\sigma)$, where $\sigma > 0$.

The interest rate in the economy is $r$. For every dollar an agent saves, the agent gets $1 + r$ in period 2. Buying one unit of consumption costs \$1 in both periods.

\begin{enumerate}
\item (6 points) Write down the budget constraints of the agents.
\item (6 points) Find the optimal level of savings for the agent as a function of $r$.
\item (6 points) Suppose that there are $N$ agents who all have wealth $W_1$ in time 1 and $0$ wealth at time 2, with the same utility function as above. What is the supply of savings at time 1?
\item (6 points) Suppose that demand for savings as a function of the interest rate is given by $D = \kappa/(1 + r)^\sigma + (1 + r)/(1 + r)^\sigma$, where $\kappa$ is a constant. Find the equilibrium interest rate in this market (solve for $r$ as a function of $N$, $W_1$ and $\kappa$.
\item (6 points) How do changes in $N$, $W_1$, $\kappa$ affect the equilibrium interest rate? Explain the intuition.
\end{enumerate}

MIT OpenCourseWare \url{https://ocw.mit.edu}. 14.01 Principles of Microeconomics, Fall 2018

\end{document}