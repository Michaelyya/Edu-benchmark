\documentclass[11pt,a4paper]{article}

\usepackage{amsmath}

\title{{Fall 2018 14.01 Problem Set 4 - Solutions}}
\begin{document}
\maketitle

\section*{Problem 1: True or False (24 points)}

\subsection*{Q1. (4 points)}
In the short and long run, a profit-maximizing firm will choose its input mix based on $MRT_S = -\frac{w}{r}$\\
Solution: False, in the short run the firm can’t choose K, so this condition may not hold. This condition only holds in the long-run when the firm can choose optimally both capital and labor.

\subsection*{Q2. (4 points)}
Long-run marginal costs can be lower or higher than short-run marginal costs, but long-run average costs can’t be higher than the short-run average costs.
Solution: The long-run average costs can’t be higher than the short-run average costs because in the long-run you can choose both inputs ( L and K) while in the short-run you can only choose L. However, the marginal costs can be either higher or lower: think for example of a production function $F (L, K) = \sqrt{LK}$. The long-run marginal costs are $MCLR = 2 \sqrt{wr}$, while the short-run marginal costs are $MCSR = 2 \frac{K\sqrt{w}}{q}$ , so for small q we will have that $MCSR (q) < MCLR (q)$ but  for large q we will have the opposite.

\subsection*{Q3. (4 points)}
In a perfectly competitive market with identical firms, a permanent positive demand shock leads to a permanent increase in the price in the long run.
Solution: False, firm entry can bring the equilibrium price back to the same level as in the initial equilibrium. As seen in lecture, under some conditions the long run supply curve is perfectly elastic so the equilibrium price is equal to the minimum average total cost.
% The latex code is cut here because it's very long.
\end{document}